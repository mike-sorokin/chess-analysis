\chapter{Background}\label{backgroundChapter}

The motivation behind solving chess puzzles, especially under time pressure, is
to improve one's pattern recognition abilities. \citet{thoughtAndChoice} showed
that chess players do not have a square-by-square recollection of the chess
board during play. Instead, they rely on interactions, and potential
interactions of pieces in a more abstract sense. This is made strikingly clear
by a professional chess player's drawing of a position from memory
(\Cref{deGrootFigure}).

\citet{thoughtAndChoice} writes, ``the pieces themselves do not appear in the
drawing: rather the lines of force that go out from them''. This heuristic
allows expert human players to quickly hone in on the correct moves
\citep{bilalic2010mechanisms} and significantly reduces their search space,
when compared to a naive brute-force search, which is the most common strategy
for chess engines. When analysing positions, even the most basic chess engines
are able to calculate the most accurate line (except some pathological cases),
but they are unable to draw similarities between different positions.

\begin{figure}[H]
  \centering
  \includegraphics[width=0.9\linewidth]{background/img/deGroot.png}
  \caption{Taken from `Thought And Choice In Chess'
  \citep{thoughtAndChoice}.}
  \label{deGrootFigure}
\end{figure}

Two sample chess positions are shown (\Cref{chess1,chess2}), featuring
\emph{back-rank checkmates} -- one of the first tactical patterns that a
beginner might learn. A castled king is forced to stay on the back-rank due to
the configuration of his men, and an opposing heavy piece exploits this
weakness by delivering checkmate. It is already non-trivial to programmatically
identify this type of tactic, as a checkmate delivered to a king on the back
rank is not necessarily a \emph{back-rank checkmate}.

Given positions similar to ones like these, how does one draw similarities
between the abstract relations of the heavy piece, king, and vulnerable back
rank?

\begin{figure}[H]
  \begin{minipage}[t]{0.475\textwidth}
    \centering
    \chessboard[setfen=6k1/5ppp/8/8/8/8/r4PPP/1R4K1 w - - 0 1]
    \caption{A trivial back-rank checkmate, White mates with
    \texttt{1.Rb8\#}.}
    \label{chess1}
  \end{minipage}
  \hspace{0.05\textwidth}
  \begin{minipage}[t]{0.475\textwidth}
    \centering
    \chessboard[setfen=6k1/5ppp/1p1Q4/p3p1B1/Pn4P1/1q6/1Pr4P/K6R w - - 1 2]
    \caption{White mates with \texttt{1.Qd8\#}.}
    \label{chess2}
  \end{minipage}
\end{figure}

\section{Chess Notation}

For many decades, the standard for chess notation has been the Algebraic
System, where, generally speaking, moves are encoded by the piece abbreviation
(\texttt{Q} for queen, \texttt{K} for king, \texttt{N} for kNight) and a
reference to the destination square, which can be any letter \texttt{a-h} and
any number \texttt{1-8} \citep{fideNotation}. This project assumes that the
reader is familiar with the basic chess rules such as piece movement patterns.

When storing chess games and positions in a computer-readable format was
required, many formats were proposed by various people, but Portable Game
Notation (PGN) and Forsyth-Edwards Notation (FEN) stuck \citep{pgnNotation}.

\subsection{Portable Game Notation (PGN)}

PGN is a specific way to write chess moves in a text file. An example, taken
from Section 2.3 of the PGN Standard \citep{pgnNotation} is shown below. Along
with some metadata at the top of the file (player names, location, etc.\@), the
chess moves are written in algebraic notation, with an incrementing move
counter. In this example, Fischer played \texttt{e4} as his first move, with
Spassky responding with \texttt{e5}. The game ended in a draw on Spassky's
\nth{43} move.

\begin{verbatim}

[Event "F/S Return Match"]
[Site "Belgrade, Serbia JUG"]
[Date "1992.11.04"]
[Round "29"]
[White "Fischer, Robert J."]
[Black "Spassky, Boris V."]
[Result "1/2-1/2"]

1. e4 e5 2. Nf3 Nc6 3. Bb5 a6 4. Ba4 Nf6 5. O-O Be7 6. Re1 b5 7. Bb3 d6 8. c3
O-O 9. h3 Nb8 10. d4 Nbd7 11. c4 c6 12. cxb5 axb5 13. Nc3 Bb7 14. Bg5 b4 15.
Nb1 h6 16. Bh4 c5 17. dxe5 Nxe4 18. Bxe7 Qxe7 19. exd6 Qf6 20. Nbd2 Nxd6 21.
Nc4 Nxc4 22. Bxc4 Nb6 23. Ne5 Rae8 24. Bxf7+ Rxf7 25. Nxf7 Rxe1+ 26. Qxe1 Kxf7
27. Qe3 Qg5 28. Qxg5 hxg5 29. b3 Ke6 30. a3 Kd6 31. axb4 cxb4 32. Ra5 Nd5 33.
f3 Bc8 34. Kf2 Bf5 35. Ra7 g6 36. Ra6+ Kc5 37. Ke1 Nf4 38. g3 Nxh3 39. Kd2 Kb5
40. Rd6 Kc5 41. Ra6 Nf2 42. g4 Bd3 43. Re6 1/2-1/2 

\end{verbatim}

The main advantages of PGN are the ease with which it can be read by a human
\citep{pgnNotation} and, given its consistent structure, the simplicity of
parsing it programmatically. This led PGN to become widely adopted and
recognised by almost all chess software. However, a downside of PGN is that it
is difficult to retrieve a position at a given move -- it must be recalculated
by tracing the moves of the game from the start.

\subsection{Forsyth-Edwards Notation (FEN)}\label{fenSection}

An alternative notation used for static chess positions is FEN. This is much
more compact than PGN, as no metadata about the game is stored other than the
state of the board. The following FEN string encodes a standard position
arising from a King's Gambit.\footnote{An incredibly aggressive, all-or-nothing
opening for White. Also the authors' weapon of choice.} The board is also shown
in \Cref{chessKGA}.

\begin{verbatim}
rnbqkbnr/pppp1ppp/8/8/4Pp2/5N2/PPPP2PP/RNBQKB1R b KQkq - 1 3
\end{verbatim}

\begin{figure}[H]
  \centering
  \chessboard[setfen=rnbqkbnr/pppp1ppp/8/8/4Pp2/5N2/PPPP2PP/RNBQKB1R b KQkq -
  1 3]
  \caption{King's Gambit Accepted, King's Knight Gambit.}
  \label{chessKGA}
\end{figure}

In FEN, pieces have their standard one-letter names, but pawns are explicitly
referred to with \texttt{P} or \texttt{p}. The number of consecutive empty
squares on a rank is denoted by an integer between \texttt{1} and \texttt{8}.
The case of the letter dictates what colour the piece is: uppercase for white,
and lowercase for black. The arrangement of pieces is stored per rank,
left-to-right, with the first appearing rank in the string corresponding to
rank 8 on the board, and forward slashes delimiting the ranks. In the given
example, the \nth{6} rank is stored as \texttt{8}, meaning there are 8
consecutive empty squares on this rank. The \nth{4} rank is stored as
\texttt{4Pp2}, meaning there are 4 empty squares, followed by a white pawn,
then a black pawn, then 2 more empty squares.

The remaining characters store other information about the board. The single
\texttt{b} means it is Black to move; the string \texttt{KQkq} means White can
castle kingside (\texttt{K}) and queenside (\texttt{Q}), and Black can castle
both ways too (\texttt{kq}). If castling was not possible, this would be
denoted as \texttt{-}. The final 3 characters are the en
passant\footnote{French for `in passing'. A special move where a pawn that
advanced two squares can be captured by an enemy pawn as if it was on the
square behind. This is an unending source of confusion for beginner players.}
file, number of halfmoves since the last pawn move or capture\footnote{Used for
the fifty move draw rule.}, and the number of full moves played.

\subsection{Bitboards}

When implementing chess programs on a computer, neither PGN or FEN are
applicable as they are difficult and slow to analyse, being string-based.
Instead, there are a number of various numerical approaches to storing a chess
position, with bitboards being one of the bigger ones. One of the first
mentions of bitboards was by \citet{bitboardsRussian}.

Described informally, a bitboard is a collection of 12 64-bit integers, where
each integer corresponds to a piece type and colour. The bits of that integer
encode the position of a piece on the board: a piece on \texttt{h1} activates
the \nth{1} (least-significant) bit. Since modern processors utilise 64 bits,
this makes bitboards very fast -- especially when using other bitmaps to check
for piece attack maps and piece interactions. In fact, bitboards are used by
\citet{stockfishBitboard}, one of the strongest chess engines today.

\section{Examples of Chess Puzzle Themes}

Some of the following positions are taken from \citet{chesscomTactics}.

\begin{figure}[H]
  \begin{minipage}{0.475\textwidth}
    \centering
    \chessboard[setfen=5r1k/4q1pp/3n2B1/1R5Q/8/7P/6P1/7K w - - 0 1]
  \end{minipage}
  \hspace{0.05\textwidth}
  \begin{minipage}{0.475\textwidth}
    \paragraph{Mate In One}White wins in one move with \texttt{1.Qxh7\#}. There
    are many variants of simple checkmates, some of which are covered below.
  \end{minipage}
\end{figure}

\begin{figure}[H]
  \begin{minipage}{0.475\textwidth}
    \centering
    \chessboard[setfen=6k1/5ppp/3r4/8/8/3q4/5PPP/R2B2K1 b - - 0 1]
  \end{minipage}
  \hspace{0.05\textwidth}
  \begin{minipage}{0.475\textwidth}
    \paragraph{Back-Rank Mate}Black wins with \texttt{1...Qxd1+ 2.Rxd1 Rxd1\#}.
    These positions occur very often when a king is castled, and the pawns
    surrounding the king on the \nth{2}/\nth{7} rank are preventing his escape
    to a different rank. A rook or queen delivers the checkmate
    single-handedly.
  \end{minipage}
\end{figure}

\begin{figure}[H]
  \begin{minipage}{0.475\textwidth}
    \centering
    \chessboard[setfen=q2r3k/6pp/8/3Q2N1/8/8/5PPP/6K1 w - - 0 1]
  \end{minipage}
  \hspace{0.05\textwidth}
  \begin{minipage}{0.475\textwidth}
    \paragraph{Smothered Mate}After \texttt{1.Nf7+ Kg8}, White wins by
    delivering a \emph{double~check} with \texttt{2.Nh6+}. Since Black is
    unable to remove both attacks on his king in one move, he is forced to move
    his king. \texttt{2...Kf8} leads to immediate mate with \texttt{3.Qf7\#},
    so Black runs to the corner with \texttt{2...Kh8}. The white queen is
    sacrificed with \texttt{3.Qg8+ Rxg8}, entombing the black king. The white
    knight delivers checkmate with \texttt{4.Nf7\#}.
  \end{minipage}
\end{figure}

\begin{figure}[H]
  \begin{minipage}{0.475\textwidth}
    \centering
    \chessboard[setfen= r2qk1nr/ppp2ppp/2np4/4p3/1b2P1b1/2NPBN2/PPP2PPP/R2QKB1R
    w KQkq - 2 6]
  \end{minipage}
  \hspace{0.05\textwidth}
  \begin{minipage}{0.475\textwidth}
    \paragraph{Absolute and Relative Pins}In this position, Black's bishops are
    pinning White's knights on \texttt{c3} and \texttt{f3}. The white knight on
    \texttt{c3} is under an \emph{absolute~pin} from the black bishop on
    \texttt{b4}, as moving it is illegal. The white knight on \texttt{f3} is
    under a \emph{relative~pin} from the other black bishop on \texttt{g4}.
    While moving this knight is legal, it is rarely done, as the bishop is then
    able to capture the white queen on \texttt{d1}.
  \end{minipage}
\end{figure}

\begin{figure}[H]
  \begin{minipage}{0.475\textwidth}
    \centering
    \chessboard[setfen=2r3k1/8/8/n1n2N2/8/8/1P6/6K1 w - - 0 1]
  \end{minipage}
  \hspace{0.05\textwidth}
  \begin{minipage}{0.475\textwidth}
    \paragraph{Fork}\emph{Fork}s are a type of \emph{double attack} and they
    come in multiple flavours, as all pieces can attack simulatenously. Perhaps
    the most common is the \emph{knight~fork}, an example of which is possible
    on the board with \texttt{1.Ne7+}, attacking both the black rook on
    \texttt{c8} and black king on \texttt{g8}. After moving the king, Black
    gives up the rook for a knight -- an unfavourable exchange.
    \\~\\
    The double pawn advance \text{1.b4} is also a \emph{fork}. Black's knights
    are attacked and even though they are able to defend each other, Black
    cannot move both to safety. One of them will be captured by the pawn.
  \end{minipage}
\end{figure}

\begin{figure}[H]
  \begin{minipage}{0.475\textwidth}
    \centering
    \chessboard[setfen=r2qkbnr/ppp2ppp/2np4/4N3/2B1P3/2N4P/PPPP1PP1/R1BbK2R w
    KQkq - 0 7]
  \end{minipage}
  \hspace{0.05\textwidth}
  \begin{minipage}{0.475\textwidth}
    \paragraph{Attack On \texttt{f2}/\texttt{f7}}The \texttt{f2} and
    \texttt{f7} squares are very vulnerable for White and Black, respectively.
    At the beginning of the game, these squares are defended only by the king.
    If neglected, a quick attack, or even a \emph{sacrifice}, can be fatal.
    \\~\\
    In this position, after sacrificing the queen, White's bishop can capture
    the weak pawn on \texttt{f7}, leading to \texttt{1.Bxf7+ Ke7 2.Nd5\#}. 
  \end{minipage}
\end{figure}


\section{Existing Work}\label{relatedWorkSection}

\subsection{Chess Board Representation}

\paragraph{CNNs For Identifying Winning Positions}\citet{chessCNN} explores
the effectiveness of neural networks to analyse whether a chess position is
winning or losing, without creating specific look-ahead algorithms. This is a
very challenging task, but as part of the analysis, the author proposed
manually encoded features to a convolutional neural network (CNN) to help
identify strong patterns within the position. 

Some of these include extra features if the opponent is in check, specifying
the squares controlled by a piece exerting a pin on a different piece, centre
control, and vulnerable squares \citep{chessCNN}. These are well known
heuristics that are often taught to beginner-intermediate chess players, and
the author claims that these additional layers are ``extremely representative
of the chess positions'', and cause the CNN to outperform a naive,
fully-connected neural network.

\paragraph{Automatic Embedding}\citet{chess2vec} investigate and report on
their results in finding $d$-dimensional vector representations of the standard
way to represent a chess board in machine learning -- a bitboard, which is
equivalent to a $8\times8\times12$ binary tensor. 

The authors make the observation that this represents each chess square as a
12-dimensional vector, which has no information about the rest of the board. By
adding a board hash, they modify the piece layers, making their vector
representations depend on the board state. With this method, \citet{chess2vec}
claim to predict $8.8\%$ of Stockfish moves correctly, which is an impressive
statistic given no chess logic or ruleset was explicitly implemented.

The results of these works served as initial inspiration for the design of the
deep learning model (\Cref{mlS21}).

\subsection{Puzzle Type and Difficulty Prediction}

\paragraph{Detecting Whether a Successful Tactic
Exists}\citet{middleGamePatterns} report their findings on a program to
identify the \emph{Greek gift sacrifice} tactic in chess
middlegames.\footnote{The phase of the chess game where most pieces are
developed and the kings are positioned away from danger.} They found success,
partly thanks to their detailed representation of the board, where each square
is represented by 71 binary attribites \citep{middleGamePatterns} to encode the
piece on the square and the possible squares which are reachable by this piece. 

These attributes were also supplemented by 59 other binary attribites which
were devised by expert chess players, and represent the more complex, but still
quantifiable relationships. These include open files, control of vulnerable
squares, and piece activity, among others.

\citet{middleGamePatterns} achieved an 87.7\% classification performance on
detecting whether a position features a successful \emph{Greek gift sacrifice}
on relatively small ($<200$ positions) datasets. 

\paragraph{Meaningful Search Trees}\citet{chessTrees} explore estimation of
difficulty of chess problems in this paper. By using a powerful chess engine,
they generate a tentative move tree of forcing or winning moves in a given
position, mirroring the process of calculating a variation for a chess player. 

By creating higher level metrics based off these trees (e.g. branching factor,
piece variety, move distances) and applying standard statistical machinery, the
authors are able to categorise the puzzles into various difficulty levels with
some success.

Their unique `meaningful search tree' \citep{chessTrees} approach to analysing
chess puzzle moves is full of potential, and is this is built upon and explored
further with the tree-based approach (\Cref{treeS12}).

\paragraph{Static and Dynamic Similarity}In their thesis, \citet{chessMotifs}
develop a method to retrieve similar chess positions among a variety of
tactically sharp examples. Building upon the naive `static' similarity (piece
locations, pawn structure, etc.\@), they implement various `dynamic' similarity
metrics (whether captures occur, attacks happen, sacrifices happen, etc.\@ in
the optimal move chain). It is well known \citep{thoughtAndChoice,
bilalic2010mechanisms} that chess players often think about positions in this
dynamic way. This is further reinforced by an example given in the thesis,
where a small static perturbation causes previously dynamically similar
positions to have nothing in common \citep{chessMotifs}.

Importantly, their results are successful: when a chess expert was asked to
identify similar positions, they chose the same ones that the program of
\cite{chessMotifs} did. While this is not a faultless evaluation metric, it
shows that this approach works. The ideas of this thesis are the foundation of
the distance function used in the tree-based approach (\Cref{treeS13}).

\section{Lichess Puzzle Database}\label{lichessPuzzlesSection}

Lichess (\url{https://lichess.org}) is a popular, open-source chess website
which often publishes the games that have been played by players of all skill
levels on it. As part of this, Lichess has published over 3.6 million rated and
tagged puzzle positions \citep{lichessPuzzles}. To generate these, 300 million
games were analysed with a powerful chess engine to find critical positions in
which a move must be played to capitalise on the opponent's mistake. These
puzzles were initially tagged to 60 manually created themes \citep{lichessXML},
which were identified by a Python script \citep{lichessTagger}.

As various users of the site solved the puzzles, and manually highlighted the
themes that they thought occured in the puzzle, the ratings and tags of the
puzzle database evolved until their current state.

\subsection{Format}

These puzzles are stored in FEN format, with a reference to the game where they
appeared. They also contain the solution as a sequence of moves, the tactics
tags \citep{lichessXML}, and the puzzle rating, along with other metadata.

An example puzzle string is shown below. This puzzle is also shown visually in
\Cref{puzzle1,puzzle2}.

\begin{verbatim}
q3k1nr/1pp1nQpp/3p4/1P2p3/4P3/B1PP1b2/B5PP/5K2 b k - 0 17,
e8d7 a2e6 d7d8 f7f8,1760,80,83,72,mate mateIn2 middlegame short,
https://lichess.org/yyznGmXs/black#34,
Italian_Game Italian_Game_Classical_Variation
\end{verbatim}

\begin{figure}[H]
  \begin{minipage}[t]{0.475\textwidth}
    \centering
    \chessboard[setfen=q3k1nr/1pp1nQpp/3p4/1P2p3/4P3/B1PP1b2/B5PP/5K2 b k -
    0 17]
    \caption{Position given by the FEN string above.}
    \label{puzzle1}
  \end{minipage}
  \hspace{0.05\textwidth}
  \begin{minipage}[t]{0.475\textwidth}
    \centering
    \chessboard[setfen=q5nr/1ppknQpp/3p4/1P2p3/4P3/B1PP1b2/B5PP/5K2 w - - 1
    18]
    \caption{Black blunders: \texttt{17...Kd7}. This position is given to
    the end user as a puzzle to be solved with forcing moves. Checkmate is
    imminent with \texttt{18.Be6+ Kd8 19.Qf8\#}.}
    \label{puzzle2}
  \end{minipage}
\end{figure}

Processing these is a trivial task with one small detail: the given FEN strings
are the state of the game right before the critical blunder is played by the
opposing site. This means the puzzle, as shown to the user, is the position
after the first move has been played. Fortunately, processing this data is made
simple with the \texttt{python-chess} library \citep{pythonChess}.

\subsection{Data Exploration}

The Lichess puzzle database has approximately 3.8 million rated and tagged
chess puzzles. Initially, they were automatically processed
\citep{lichessTagger}, but were then refined with user feedback
\citep{lichessPuzzles}. This also allowed the puzzles to obtain a rating, which
is indicative of its difficulty \citep{lichessPuzzles}.

Overall, there are 60 various puzzle themes \citep{lichessXML}.
\Cref{dataThemeCounts} shows the counts of each theme in all of the contained
puzzles. It should be noted that some themes are mutually exclusive (a
checkmate puzzle cannot be both \emph{mate-in-one} and \emph{mate-in-two}).

The most common puzzle themes are the most general ones -- specifically
\texttt{short}, \texttt{middlegame}, and \texttt{endgame}. There is a lot of
variation in how frequent the various patterns are, which is a natural
consequence of the game.

\begin{figure}
  \centering
  \includegraphics[width=0.9\linewidth]{project/img/puzzle_theme_counts.png}
  \caption{Frequency of puzzle themes in the Lichess puzzle database
  \citep{lichessPuzzles}.}
  \label{dataThemeCounts}
\end{figure}

\Cref{dataHistogram} shows the distribution of ratings across the chess
puzzles. These ratings are only appropriate within this dataset, and cannot be
compared to ratings of puzzles on other chess websites. The puzzle ratings are
symmetric about the mean, and this is of course a result of the Glicko2 rating
system, developed by \citet{glicko}.

When analysing the rating distribution by theme, an expected behaviour occurs.
Some chess tactics patterns are considerably simpler to spot, meaning a weaker
player is able to solve the puzzle with that theme. Therefore some puzzle
themes, \emph{back-rank mate} (\Cref{puzzle3}), for example, have lower-rated
puzzles when compared to puzzles featuring a \emph{trapped piece}
(\Cref{puzzle4}) or \emph{defensive move}.\footnote{Notoriously difficult for
humans, who are usually much better at aggression than
defense.\footnotemark}\footnotetext{In chess.}

\begin{figure}[H]
  \begin{minipage}[t]{0.475\textwidth}
    \centering
    \includegraphics[width=\textwidth]{project/img/puzzle_histogram.png}
    \caption{Distribution of Lichess puzzle ratings.}
    \label{dataHistogram}
  \end{minipage}
  \hspace{0.05\textwidth}
  \begin{minipage}[t]{0.475\textwidth}
    \centering
    \includegraphics[width=\textwidth]
    {project/img/puzzle_theme_histograms.png}
    \caption{Distribution of Lichess puzzle ratings with a specific theme.}
    \label{dataThemeHistogram}
  \end{minipage}
\end{figure}

\begin{figure}[H]
  \begin{minipage}[t]{0.475\textwidth}
    \centering
    \chessboard[setfen=6k1/5ppp/r1p5/p1n1rP2/8/2P2N1P/2P3P1/3R2K1 w - - 0
    22]
    \caption{\textbf{Kenan2345 -- gandie}, lichess.org Blitz game, move 22. 
    Black loses to \texttt{22.Rd8+}.}
    \label{puzzle3}
  \end{minipage}
  \hspace{0.05\textwidth}
  \begin{minipage}[t]{0.475\textwidth}
    \centering
    \chessboard[setfen=2rq1rk1/7p/1n4pb/1R2Q3/pPpP1P2/P1B5/3N2PP/2R3K1 b -
    - 0 31]
    \caption{\textbf{mhabib -- Sarg0n}, lichess.org Blitz game, move 31.
    White loses his queen after \texttt{31...Re8}, as the queen has no
    safe squares to escape to.}

    \label{puzzle4}
  \end{minipage}
\end{figure}

