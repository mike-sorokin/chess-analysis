\chapter{Evaluation}\label{evalChapter}

After explaining the two approachs in the previous two chapters, they are now
compared on the tasks outlined at the beginning of the report: how good are
these at predicting the themes in a puzzle (\Cref{evalS1}), and how good are
they at predicting the difficulty of a puzzle (\Cref{evalS2})?

Each of the methods is evaluated in greater detail (\Cref{evalS3,evalS4}),
focusing on their individual advantages and disadvantages
(\Cref{evalS31,evalS41}).

Finally, each method has a unique application that the other does not: a game
review for the deep learning transformer approach (\Cref{evalS32}) and a
similarity comparison for the tree-based approach (\Cref{evalS42}). These are
discussed and judged based on their own merits.


\section{Prediction of Puzzle Themes}\label{evalS1}

Both approaches (\Cref{mlChapter,treeChapter}) are compared with the unseen
test set. The deep learning approach produced a model which can be tested
against the entire test set, while the $k$-NN method (\Cref{treeS3}) is used to
evaluate the tree-based approach. Due to its computational complexity
(\Cref{treeS31}), $k$-NN uses a subset of 100,000 of its training set and is
evaluated against a random selection of 2,000 puzzles form the test set. This,
of course, increases variance, but is unavoidable without great investments in
time and processing power.

Per-label precision, recall, and F1-scores are considered for both approaches
(\Cref{labelTable}). Additionally, the averaged statistics are also shown. In
all averaging formats (micro, macro, weighted, sample), the deep learning
transformer model outperforms the tree-based $k$-NN approach, but it is
surprisingly close. 

The macro average (unweighted mean of the statistics) suffers greatly for both
approaches. This is due to the heavily unbalanced dataset
(\Cref{lichessDataExpl}), so the weighted average (more common themes have a
higher weight) is more appropriate.

The theme breakdown, despite being difficult to parse, shows some interesting
differences between the models' performances. All themes discussed in this
section are highlighted in \Cref{labelTable} for ease of readability. The
tree-based algorithm, which uses Stockfish for tree generation
(\Cref{treeS12}), is perfect at identifying \texttt{mateIn1} and almost perfect
at \texttt{mateIn2}. This is likely because mating patterns are quite common
and the meaningful move trees constructed for them are almost identical. For
this same reason, the tree-based model outperforms the deep learning
transformer model at \texttt{oneMove} and \texttt{short} label prediction.

On the other hand, the deep learning transformer model is almost completely
accurate at predicting \texttt{middlegame} and \texttt{endgame} labels, along
with the various types of endgame: \texttt{bishopEndgame},
\texttt{knightEndgame}, \texttt{pawnEndgame}, \texttt{queenEndgame},
\texttt{queenRookEndgame}, and \texttt{rookEndgame}. This is definitely a
result of the model being designed in a way where it can access the whole board
at once, and an absence of many pieces (what is a giveaway that a position is
in an endgame), is trivial for it to detect.


\begin{table}[H]
  \centering
  \begin{adjustbox}{width=\textwidth}
    \begin{tabular}{l|rrrr|rrrr}
      & \multicolumn{4}{c}{Tree-Based $k$-NN} & \multicolumn{4}{c}{Deep Learning Transformer} \\
      Theme & Precision & Recall & F1-score & Support & Precision & Recall & F1-score & Support \\
      \hline

      advancedPawn&0.7660&0.3600&0.4898&100&0.7044&0.4462&0.5463&43215\\
      advantage&0.5693&0.4091&0.4761&572&0.5966&0.5898&0.5932&235529\\
      anastasiaMate&1.0000&0.3333&0.5000&3&0.8554&0.5338&0.6574&798\\
      arabianMate&0.0000&0.0000&0.0000&1&0.7858&0.5842&0.6702&760\\
      attackingF2F7&0.8333&0.5556&0.6667&9&0.8693&0.7412&0.8002&4657\\
      attraction&0.9697&0.4156&0.5818&77&0.6858&0.1767&0.2810&26566\\
      backRankMate&0.6607&0.5692&0.6116&65&0.8236&0.8427&0.8330&24332\\
      \rowcolor{lightgray} bishopEndgame&0.7778&0.4667&0.5833&15&0.9622&0.9774&0.9697&9476\\
      bodenMate&0.0000&0.0000&0.0000&2&0.8045&0.5619&0.6617&315\\
      capturingDefender&0.0000&0.0000&0.0000&13&0.7120&0.0209&0.0406&6505\\
      castling&0.0000&0.0000&0.0000&2&0.5000&0.0071&0.0141&421\\
      clearance&0.5000&0.0400&0.0741&25&0.7768&0.0320&0.0616&10641\\
      crushing&0.7094&0.6704&0.6893&892&0.7223&0.5972&0.6538&337162\\
      defensiveMove&0.5000&0.1121&0.1832&107&0.5340&0.0462&0.0850&47212\\
      deflection&0.7857&0.1341&0.2292&82&0.7243&0.0940&0.1664&32867\\
      discoveredAttack&0.7778&0.2823&0.4142&124&0.6760&0.2641&0.3799&43352\\
      doubleBishopMate&0.0000&0.0000&0.0000&0&0.7907&0.5795&0.6689&352\\
      doubleCheck&0.0000&0.0000&0.0000&8&0.7518&0.2327&0.3554&3670\\
      dovetailMate&0.0000&0.0000&0.0000&2&0.7222&0.0323&0.0619&402\\
      enPassant&0.0000&0.0000&0.0000&0&0.7333&0.0189&0.0368&1165\\
      \rowcolor{lightgray} endgame&0.8036&0.7412&0.7711&966&0.9984&0.9999&0.9992&368846\\
      equality&0.0000&0.0000&0.0000&23&1.0000&0.0001&0.0002&8503\\
      exposedKing&0.0000&0.0000&0.0000&54&0.3561&0.1114&0.1697&21239\\
      fork&0.7500&0.5445&0.6309&281&0.7666&0.4015&0.5270&111331\\
      hangingPiece&0.6538&0.2073&0.3148&82&0.6806&0.1650&0.2657&33331\\
      hookMate&0.0000&0.0000&0.0000&5&0.6887&0.4616&0.5527&1107\\
      interference&0.0000&0.0000&0.0000&9&0.6250&0.0016&0.0032&3129\\
      intermezzo&1.0000&0.0357&0.0690&28&0.7517&0.0402&0.0763&11221\\
      kingsideAttack&0.7742&0.3357&0.4683&143&0.7146&0.6082&0.6571&61522\\
      \rowcolor{lightgray} knightEndgame&0.7000&0.4667&0.5600&15&0.9319&0.9944&0.9621&5887\\
      long&0.6809&0.1932&0.3009&497&0.5746&0.0754&0.1333&193731\\
      master&0.0000&0.0000&0.0000&148&0.2500&0.0001&0.0001&54003\\
      masterVsMaster&0.0000&0.0000&0.0000&11&0.0000&0.0000&0.0000&4967\\
      mate&0.9783&0.9688&0.9735&512&0.8210&0.7731&0.7963&198507\\
      \rowcolor{lightgray} mateIn1&1.0000&1.0000&1.0000&219&0.7483&0.6803&0.7127&80324\\
      \rowcolor{lightgray} mateIn2&0.9773&1.0000&0.9885&215&0.7727&0.5100&0.6145&91787\\
      mateIn3&0.9388&0.6866&0.7931&67&0.7096&0.2987&0.4204&22611\\
      mateIn4&0.6667&0.2500&0.3636&8&0.5562&0.0322&0.0608&3076\\
      mateIn5&0.0000&0.0000&0.0000&3&0.9062&0.0409&0.0783&709\\
      \rowcolor{lightgray} middlegame&0.7285&0.7293&0.7289&931&0.9919&0.9854&0.9886&377218\\
      \rowcolor{lightgray} oneMove&1.0000&0.8939&0.9440&245&0.7566&0.6041&0.6718&89681\\
      opening&0.5909&0.1262&0.2080&103&0.8873&0.9419&0.9138&44379\\
      \rowcolor{lightgray} pawnEndgame&0.7846&0.9623&0.8644&53&0.9742&0.9998&0.9868&22943\\
      pin&0.8261&0.1397&0.2390&136&0.6821&0.2167&0.3288&49568\\
      promotion&1.0000&0.0968&0.1765&31&0.6591&0.3415&0.4499&16317\\
      \rowcolor{lightgray} queenEndgame&1.0000&0.2500&0.4000&16&0.9748&0.9915&0.9830&7282\\
      \rowcolor{lightgray} queenRookEndgame&0.0000&0.0000&0.0000&17&0.9482&0.9170&0.9324&5171\\
      queensideAttack&0.7143&0.1471&0.2439&34&0.5973&0.4695&0.5257&10552\\
      quietMove&0.9231&0.1579&0.2697&76&0.5908&0.1497&0.2388&31019\\
      \rowcolor{lightgray} rookEndgame&0.7400&0.3814&0.5034&97&0.9759&0.9893&0.9826&37058\\
      sacrifice&0.8571&0.3947&0.5405&152&0.7042&0.2853&0.4061&55181\\
      \rowcolor{lightgray} short&0.7341&0.8385&0.7829&1090&0.6810&0.6802&0.6806&434161\\
      skewer&0.6667&0.2927&0.4068&41&0.7135&0.2891&0.4115&17291\\
      smotheredMate&0.5000&0.2500&0.3333&4&0.8920&0.8392&0.8648&2264\\
      superGM&0.0000&0.0000&0.0000&0&0.0000&0.0000&0.0000&462\\
      trappedPiece&1.0000&0.1481&0.2581&27&0.6613&0.1431&0.2352&11135\\
      underPromotion&0.0000&0.0000&0.0000&0&0.0000&0.0000&0.0000&140\\
      veryLong&0.7500&0.0180&0.0351&167&0.6098&0.0084&0.0166&62128\\
      xRayAttack&0.0000&0.0000&0.0000&5&0.7980&0.2401&0.3691&2699\\
      zugzwang&0.4444&0.2000&0.2759&20&0.7079&0.3097&0.4308&6604\\
      \hline
      micro~avg&0.7774&0.5869&0.6689&8630&\textbf{0.8025}&\textbf{0.6040}&\textbf{0.6893}&3388481\\
      macro~avg&0.5339&0.2801&0.3324&8630&\textbf{0.7065}&\textbf{0.3996}&\textbf{0.4497}&3388481\\
      weighted~avg&0.7416&0.5869&0.6211&8630&\textbf{0.7600}&\textbf{0.6040}&\textbf{0.6388}&3388481\\
      samples~avg&0.7501&0.5952&0.6488&8630&\textbf{0.8063}&\textbf{0.6181}&\textbf{0.6734}&3388481\\
    \end{tabular}
  \end{adjustbox}
  \caption{Breakdown of the node distance function for meaningful move trees}
  \label{labelTable}
\end{table}


\section{Prediction of Puzzle Difficulty}\label{evalS2}


\section{Deep Learning Approach}\label{evalS3}

\subsection{Strengths and Weaknesses}\label{evalS31}

The advantages of this approach are clearly the better performance
(\Cref{labelTable}) over the tree-based approach. After a relatively long
training period, the models produced by this method can analyse new positions
almost instantly. As the output of the model is a number in the range $(0, 1)$,
a sense of `confidence' can be easily understood also, which is considerably
more difficult in the tree-based approach with $k$-NN.

The biggest downside is the trade-off of having fast inference: slow training
time. This, combined with the need to search for hyperparameters (\Cref{mlS22})
makes this method quite time and resource intensive. 

\subsection{Game Review}\label{evalS32}

An interesting unintended feature of this approach is that it can, of course,
predict puzzle themes for any chess position. Since inference time is so fast,
it is possible to iterate through any chess game's moves and analyse those
positions. For this example, a game played by the authors was analysed, and the
interesting positions, along with their labels, are shown below.

Interestingly, the starting position (\Cref{alapin1}) has the labels
\texttt{crushing}, \texttt{hangingPiece}, and \texttt{oneMove}. The model does
correctly predict \texttt{opening}. 

After playing a strange sequence of pawn moves (\Cref{alapin2}), Black gets a
losing position, where White has a classic \emph{attack on f7} tactic. Indeed,
White's best move in this position is a \emph{bishop sacrifice},
\texttt{9.Bxf7+ Kxf7}, which walks into a knight check: \texttt{10.Ng5+ Ke8
11.Qxg4}. Right before the \emph{bishop sacrifice} was played, the model
correctly predicts \texttt{attackingF2F7}, \texttt{attraction} (the black king
is attracted onto a square where it can be checked by the knight) and
\texttt{sacrifice}. This is the only position throughout the entire game where
the model predicts all of these themes. As a side note, the position
immediately prior to \texttt{10.Ng5+} is labelled \texttt{discoveredAttack}. It
is also at this point that the model stops predicting \texttt{opening} and
begins predicting \texttt{middlegame}.

This attack leads to what seems to be a mating position (\Cref{alapin3}).
Despite this, White does not have a checkmate. The player with the white pieces
testifies that he thought the position must have a forced mate, and the model
agrees, predicting \texttt{mate} and \texttt{mateIn1}. Both are wrong.

After many moves, White fails to convert his advantage (\Cref{alapin4}), and
ends up in an equal position. After the rook trade, \texttt{30...Rxe1+ 31.Rxe1
Bxd4}, the model switches its predictions from \texttt{middlegame} to
\texttt{endgame}. No more sharp positions appear for the rest of the game and
the model's predictions are equally uninteresting.

\begin{figure}[H]
  \begin{minipage}[t]{0.475\textwidth}
    \centering
    \chessboard[setfen=rnbqkbnr/pppppppp/8/8/8/8/PPPPPPPP/RNBQKBNR w KQkq - 0 1]
    \caption{The starting position.}
    \label{alapin1}
  \end{minipage}
  \hspace{0.05\textwidth}
  \begin{minipage}[t]{0.475\textwidth}
    \centering
    \chessboard[setfen=2rqkbnr/3npppp/p2p4/1ppB4/3PP1b1/2P2N2/PP3PPP/RNBQ1RK1 w k - 0 9]
    \caption{White has a winning advantage.}
    \label{alapin2}
  \end{minipage}
\end{figure}

\begin{figure}[H]
  \begin{minipage}[t]{0.475\textwidth}
    \centering
    \chessboard[setfen=2rqkb1r/4p1pp/pn1pQn2/1pp3N1/3PP3/2P5/PP3PPP/RNB2RK1 w - - 3 13]
    \caption{Despite looking lethal for Black, White does not have a forced mate.}
    \label{alapin3}
  \end{minipage}
  \hspace{0.05\textwidth}
  \begin{minipage}[t]{0.475\textwidth}
    \centering
    \chessboard[setfen=1r6/3kr1b1/3p3p/3n2p1/1p1P4/6N1/P2B1PPP/1R2R1K1 b - - 4 30]
    \caption{After the rook trade, White loses the \texttt{d4} pawn and loses the game soon after.}
    \label{alapin4}
  \end{minipage}
\end{figure}


\section{Tree-Based Approach}\label{evalS4}

\subsection{Strengths and Weaknesses}\label{evalS41}

The biggest weakness of the tree-based approach -- using $k$-NN to predict
labels and difficulty -- is the inference time. This has had large consequences
previously in the project, and this is an unavoidable result of using a
relatively expensive and non-vectorisable distance function. This downside
meant that the method could not be evaluated fully, as it was time prohibitive
to analyse the entire test set while using the entire training set.

Another downside is the sensitivity to parameters (\Cref{treeS13}). Different
parameters in the distance function define what it means for a puzzle to be
similar, and this, of course, can also vary drastically between chess players.
The other set of parameters that greatly influences the performance is the tree
generation parameters (\Cref{treeS12}). These are again non-trivial to
optimise, as it takes a long time to regenerate the meaningful search trees.
These parameters also change from person to person, especially the arbitrary
move consideration boundary, which was kept at 100 centipawns in this project.
It is very likely that a higher skilled player would not even consider
low-performing moves when comparing if two puzzles are similar, whereas a lower
rated player would calculate more variations deeper as he is unable to
understand the strength of a position with intuition alone.

However, within this weakness also lies the main strength of this method. This
tunability means chess players of all skills can benefit from the distance
function, as they can tune it to their performance.

Of course, this also means this method of puzzle comparison is much more
explainable than the black-box deep learning approach. It is completely
possible to view a puzzle's meaningful move tree, and to reason about why a
certain puzzle is said to be close to another, or said to have a certain label
or difficulty.

\subsection{Puzzle Similarity}\label{evalS42}

Grouping puzzles (\Cref{treeS2}) and ranking puzzles by similarity
(\Cref{treeS21}) is unique to this method. Performance of clustering puzzles
has been shown previously 

