\chapter{Introduction}

Chess has long been analysed and enjoyed by both amateurs and professionals
alike, and with the birth of computer science, an algorithm, along with a
powerful enough computer to run it on, has been hunted for since at least
the 1950s.\cite{shannon} In 1997, Garry Kasparov, the strongest chess
Grandmaster at the time, was defeated by Deep Blue,\cite{deepBlue} and
in the world of chess, humans were displaced by machines forever. 

Despite this, chess remains a popular pastime for many people, with the largest
online chess website, \url{chess.com}, having over 160,000,000 total members
\cite{chesscomMembers}. Whilst chess engines compete at the superhuman level
with cutting-edge techniques, many average chess players refine their ability
with study of previous positions -- chess puzzles -- where the goal is to find
the critical move to punish the opponent's mistake and gain a decisive
advantage, or win the game immediately via checkmate. 

These puzzles seek to train the `tactics' element of one's chess ability, and
they can be categorised by the theme that occurs within them; some of these
include \emph{fork}ing enemy pieces with a knight, a well-known bishop
sacrifice known as the \emph{Greek gift sacrifice}, and many more, in addition
to their combinations.\cite{chessPatterns} This allows players to identify
which types of patterns they are often overlooking in real games, and to
selectively train to detect those. 

It is surprising that this categorisation is most often done manually, be it
via community voting,\cite{lichessPuzzles} or a list of rules that the puzzle
must satisfy to be classed one way or another.\cite{lichessTagger} This, along
with the infamous saying that `chess is 99\% tactics', suggests that there is a
need to explore this problem and provide in-depth categorisation of puzzles
(beyond what is currently the norm).

Chess Grandmasters often talk about the `feeling' of a position -- is it
possible to quantify this?

\section{Challenges}

(This section is blank... for now)

\section{Contributions}

(This one too)

