\chapter{Introduction}

Chess has long been analysed and enjoyed by both amateurs and professionals
alike, and with the birth of computer science, an algorithm, along with a
powerful enough computer to run it on, has been hunted for since at least the
1950s \citep{shannon}. In 1997, Garry Kasparov, the strongest chess Grandmaster
at the time, was defeated by Deep Blue \citep{deepBlue}, and in the world of
chess, humans were displaced by machines forever. 

Despite this, chess remains a popular pastime for many people, with the largest
online chess website, \url{chess.com}, having over 160,000,000 total members
\citep{chesscomMembers}. Whilst chess engines compete at the superhuman level
with cutting-edge techniques, many average chess players refine their ability
with study of previous positions -- chess puzzles -- where the goal is to find
the critical move to punish the opponent's mistake and gain a decisive
advantage, or win the game immediately via checkmate. 

These puzzles seek to train the `tactics' element of one's chess ability, and
they can be categorised by the theme that occurs within them; some of these
include \emph{fork}ing enemy pieces with a knight, a well-known bishop
sacrifice known as the \emph{Greek gift sacrifice}, and many more, in addition
to their combinations \citep{chessPatterns}. This allows players to identify
which types of patterns they are often overlooking in real games, and to
selectively train to detect those. 

It is surprising that this categorisation is most often done manually, be it
via community voting \citep{lichessPuzzles}, or a list of rules that the puzzle
must satisfy to be classed one way or another \citep{lichessTagger}. This,
along with Teichmann's infamous saying `chess is 99\% tactics', suggests that
there is a need to explore this problem and provide in-depth categorisation of
puzzles (beyond what is currently the norm).

Chess Grandmasters often talk about the `feeling' of a position -- is it
possible to quantify this?

\section{Objectives}

The goal of this project is to explore novel ways to design and implement chess
puzzle classifiers. It should improve upon existing solutions by

\begin{enumerate}
    \item Better analysing the tactical themes which occur in the puzzle.
    \item Creating a way to calculate a `distance' between chess puzzles to 
    group puzzles by theme and difficulty similarity.
    \item Calculating the overall difficulty of a problem.
\end{enumerate}

\section{Report Structure}

Initially, the background of chess programming is covered
(\Cref{backgroundChapter}), including notation and programmatic
representation of chess. Some pre-existing work is touched upon
(\Cref{relatedWorkSection}), including the Lichess puzzle database.

The problem is first tackled in the context of deep learning
(\Cref{mlChapter}). By treating individual chess squares as tokens and
applying the infamous transformer, a model to predict puzzle labels is designed
and trained.

Following this, a different approach is taken. With the help of search trees
(\Cref{treeChapter}), the project attempts to capture how a human might
compare puzzles. This leads to a notion of distance between puzzles and lends
itself to unsupervised clustering.

Both of these methods are then compared, contrasted, and evaluated against a
common metric (\Cref{evalChapter}). Advantages and disadvantages of each
are highlighted.

The project concludes (\Cref{concChapter}) with a summary of the
technical achievements, challenges, and possible future directions. Comparisons
are drawn between the findings of this project and existing work in the field.
