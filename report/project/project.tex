\chapter{Planning}
Note: this section will be should be removed after the interim report!
\section{Project plan}

\subsection{Research more literature}
There are still a number of papers that attempt to come up with new ways to
algorithmically analyse chess positions and these should also be reviewed. There
is also a very large amount of papers to do with psychology of professional chess
players, and the biggest conclusion of these, the fact that chess players often
analyse the piece patterns and links, not the squares \cite{thoughtAndChoice}, 
has been the backbone of many papers. It's possible that there are still untapped
findings that could be explored by this project. 

\subsection{Decide on deterministic/graph-based vs CNN approach (or both?)}
So far, I have come across 2 main ways in which previous work has dealt with
problems similar to this one. The first is a way to logically encode piece
relationships \cite{chessPatterns}, \cite{bilalic2010mechanisms}, \cite{chessLanguage}, 
\cite{cql}, \cite{lichessTagger}, which can prove to be difficult without manually
encoding patterns. The second is a machine learning approach, usually with CNNs,
since they tend to effectively `identify patterns' \cite{chessCNN}, \cite{chessKernel},
\cite{middlegamePatterns}. Since my project concerts itself only with a stationary 
position, without accounting for lookforward moves, it's possible that some sort
of encoding can be use (in fact, there is a paper called Chess2Vec \cite{chess2vec}
which has attempted to predict chess moves -- not analyse positions).


\subsection{Implement a prototype to categorise puzzles by lichess label and difficulty, using
their puzzle tagger as a helper}
A well written prototype should be created to implement the algorithm described.
The implementation language depends on the algorithm chosen, but my initial guess
is that any logical solution would lend itself to Haskell (or another FP language),
whilst the most natural language for any machine learning approach is python (almost
definitely with the pytorch library, as I have the most experience with it). 

A soft requirement for a language for this is to have a well-made chess library
to analyse moves, pieces, etc. Implementing one from scratch is hopefully not
too difficult, but reinventing the wheel is unnecessary.

\subsection{Given a set of positions, attempt to cluster them into labels. The labels
are either manual or automatic(?) or even absent(? unsupervised?)}
This is part of the implementation and is in essence, using the output of the 
written program to analyse a set of chess problems.

\subsection{Experiment with the other approach, contrast and compare}
Time permitting, the other approach (either graph-based or CNN) could be analysed.
The outputs, advantages, disadvantages, etc. of the 2 methods should be compared
and contrasted here -- it's possible that one of them works much better than the other.


\section{Evaluation plan}
\subsection{Quantitative}
The simplest metric to check if the solution is correct would be to compare
against lichess's 3.6million database. It is possible there exist other
chess position databases (this should be investigated) to compare to. For
obvious reasons, this will be split into training/test (if the algorithm requires
training on positions). Lichess's puzzle tags can be treated as a multi-label
classification algorithm, and as long as the project produces a similar output,
this can be used. The labels and difficulty can also be used to create a basic
notion of distance based on lichess's classification, which should then be compared
with the project's results.

Some papers researched in the background section (and possibly ones that are yet
to be read) have attempted a similar problem with chess position analysis. The
results of this project can be then compared against their solution, hopefully
on the same benchmark.

\subsection{Qualitative}
Multiple positions can be designed/retrieved from a database and analysed by a
chess player. Of course, the player's skill dictates the complexity of positions
which they are able to sufficiently analyse and compare, but with the help of
modern chess engines, it should be possible for an intermediate player who is
aware of tactic and puzzle classifications to draw conclusions. These could then
be compared to the predictions of the program implemented by the project and 
conclusions can then be drawn. 