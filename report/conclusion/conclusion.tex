\chapter{Conclusion}\label{concChapter}

\section{Related Work}\label{concWork}

\subsection{A Language For Encoding Piece Relationships}

\citet{chessLanguage} describe a language to search across chess positions. The
main features of this language are descriptions for a chess piece
attacking/defending another, attacking/defending a square, being located at a
square, a square not being available for the enemy king, and the structure of
white/black pawns on the board.

With their novel language, they are able to search a chess database for a
pre-determined pattern, such as the \emph{Greek gift sacrifice}, defined in the
language as ``\texttt{kg8, pf7, pg7, B(ph7), Nf3, Qd1, Pe5}''. With inspiration
from the FEN notation, this string corresponds to a black king on g8
(\texttt{kg8}), black pawns on f7, g7; a white bishop attacking a black pawn on
h7 (\texttt{B(ph7)}), a white knight on f3 (ready to deliver a check with
\texttt{Ng5+}, a common motif in \emph{Greek gift sacrifices}), a white queen
on d1 (\texttt{Qd1}), and finally, a white pawn on e5 (\texttt{Pe5}),
dislodging the usual black knight on f6.

This returns positions such as the one in Figure \ref{chess3}. After
\texttt{16...Be7 17.O-O}, Black blundered with \texttt{17...Bxa3??}, after
which, a \emph{Greek gift sacrifice} (\texttt{18.Bxh7!}, shown in Figure
\ref{chess4}) was made, eventually leading to a win for White.

\begin{figure}[H]
    \begin{minipage}{0.475\textwidth}
        \centering
        \chessboard[setfen=r1b2rk1/qp3ppp/p1n1pb2/4P3/3P4/P1BB1N2/5PPP/1R1QK2R
        b K - 0 16]
        \caption{\textbf{Pirc, V -- Porreca, G}, YUG-ITA m 1953, move 16.}
        \label{chess3}
    \end{minipage}
    \hspace{0.05\textwidth}
    \begin{minipage}{0.475\textwidth}
        \centering
        \chessboard[setfen=r1b2rk1/qp3ppB/p1n1p3/4P3/3P4/b1B2N2/5PPP/1R1Q1RK1 b
        - - 0 18]
        \caption{\textbf{Pirc, V -- Porreca, G}, YUG-ITA m 1953, move 18. Black
        resigned after 6 moves.}
        \label{chess4}
    \end{minipage}
\end{figure}

Their language is also able to deal with some light variations, as it is able
to identify the games shown in Figures \ref{chess5}, \ref{chess6}. In both of
these positions, White has the amazing move \texttt{1.Qh6+!}, following with
\texttt{2.Rh8\#} if \texttt{1...Kxh6}, and either \texttt{2.Rf7\#} or
\texttt{2.Rb7+} (leading to a quick mate) if \texttt{1...gxh6}. 

This pattern, whilst very rare, is undeniably identical between the 2 games.
The unavailability of the \texttt{g6} square to the enemy king, combined with
the harmony of White's pieces leads to the same tactic in both games.

\begin{figure}[H]
    \begin{minipage}{0.475\textwidth}
        \centering
        \chessboard[setfen=2R5/4bppk/1p1p4/5R1P/4PQ2/5P2/r4q1P/7K w - - 5 50]
        \caption{\textbf{Carlsen, M -- Karjakin S}, World Chess Championship
        2016, move 50.}
        \label{chess5}
    \end{minipage}
    \hspace{0.05\textwidth}
    \begin{minipage}{0.475\textwidth}
        \centering
        \chessboard[setfen=5R2/bp4pk/2n3p1/P7/P1q3bP/6P1/3Q3K/1R6 w - - 1 32]
        \caption{\textbf{Popov, N -- Novopashin, A}, URS-ch otbor 1979, move
        32.}
        \label{chess6}
    \end{minipage}
\end{figure}

The work of \citet{chessLanguage} is a promising proof of concept that shows
the power of a language that allows to specify piece relationships on a more
abstract level than previously possible. The biggest drawback of their
solution, as mentioned by the authors, is the fact that this language still
requires an expert with pre-existing extensive knowledge to encode the tactics
into their language.

This work is quite similar to this project's tree-based puzzle analysis
(\Cref{treeChapter}). Tree-based puzzle analysis, with a distance function, is
very computationally expensive due to the need to analyse game positions with a
chess engine. The language developed by \citet{chessLanguage} is far more
performant, which makes it much more accessible. The downside, of course, is
the need for domain knowledge. 


\subsection{CQL: Chess Query Language}

The Chess Query Language (CQL), invented by \citet{cql}, is another
implementation of an advanced way to find chess positions in a given database.
Since its inception in 2004, it has grown and is able to support very powerful,
sometimes esoteric, queries to find predefined patterns.

An example of such a query is provided on the CQL website \citep{cqlSmothered},
and is shown in Figure \ref{cql} for reference. In this query, \texttt{btm}
means `black-to-move` and \texttt{mate} means checkmate is played. This
language is incredibly powerful and terse, as it allows specifying complicated
piece relationships and supports quality-of-life features such as matching
mirror positions or reversed-colour positions.

\begin{figure}[H]
    \centering
    \includegraphics[width=0.45\linewidth]{background/img/cql.png}
    \caption{A CQL query to find positions where smothered mate occured.}
    \label{cql}
\end{figure}

In addition to Costeff's CQL, there exists a from-scratch clone of CQL6
\citep{cqli} which includes extra features and supports other chess variants.

This is a very powerful tool, but suffers from the same drawback as the work of
\citet{chessLanguage}: it requires extensive knowledge to use effectively. It
has been shown by \citet{cql} that CQL can support incredibly niche and complex
tactical patterns. It is possible that an expert, armed with the unsupervised
clustering (\Cref{treeS2}) and this language, can create efficient queries to
find critical chess positions.

\subsection{The Chess Transformer}\label{chessTransformerSection}

\citet{chessTransformer} demonstrate the ability of transformers to learn the
rules of chess and complex gameplay by analysing PGN games with a fine-tuned
GPT-2 transformer. By treating PGN games as a sequence of natural language
words, the authors show successful results and are able to generate new games
without specifying chess rules.

A key downside of their work is that their model generates illegal moves, which
have to be filtered manually using a chess library \citep{chessTransformer}.
This `hallucination' effect is a downside of using transformers, and other
generative techniques, in chess. 

This work is very similar to this project's work with transformers
(\Cref{mlChapter}). The key difference is that the work by
\citet{chessTransformer} operates on sequences of chess moves, while the work
in this project uses a transformer encoder to learn relationships between the
chess squares in a static position. It is unlikely that adding the extra
complexity of learning PGN outweighs the benefits, especially given the task of
this project. However, along with this project's work on transformers, the work
of \citet{chessTransformer} shows that there is potential in applying the
transformer architecture to the logical structure of chess.

\section{Closing Remarks}

At the be

\section{Future Work}
