\documentclass[a4paper, twoside]{report}

%% Language and font encodings
\usepackage[english]{babel}
\usepackage[utf8x]{inputenc}
\usepackage[T1]{fontenc}

%% Sets page size and margins
\usepackage[a4paper,top=3cm,bottom=2cm,left=3cm,right=3cm,marginparwidth=1.75cm]{geometry}

%% Useful packages
\usepackage{adjustbox}
\usepackage{amsmath}
\usepackage{colortbl}
\usepackage{fancyhdr}
\usepackage{float}
\usepackage{graphicx}
\usepackage{listings}
\usepackage{url}
\usepackage{xcolor}
\usepackage{xskak}
\usepackage[super]{nth}
\usepackage[square]{natbib}
\usepackage[colorinlistoftodos]{todonotes}
\usepackage[parfill]{parskip}
\usepackage{hyperref}
\usepackage[nameinlink]{cleveref}

\title{Chess Puzzle Analysis}
\author{Mihhail Sorokin}
% Update supervisor and other title stuff in title/title.tex


\fancypagestyle{thesis}{
  \fancyhf{}
  \fancyhead[L]{\nouppercase{\leftmark}}
  \fancyhead[R]{\nouppercase{\rightmark}}
  \fancyfoot[LE,RO]{\thepage}
}

\definecolor[named]{ACMPurple}{cmyk}{0.55,1,0,0.15}
\definecolor[named]{ACMDarkBlue}{cmyk}{1,0.58,0,0.21}
\hypersetup{colorlinks,
linkcolor=ACMPurple,
citecolor=ACMPurple,
urlcolor=ACMDarkBlue,
filecolor=ACMDarkBlue
}

\widowpenalty10000
\clubpenalty10000

\pagestyle{thesis}

\begin{document}
\input{title/title.tex}


\begin{abstract}

  Chess, the most popular board game of all time, is a game full of strategy,
  patterns, and `tactics'. Tactics are sequences of moves which occur across
  many games and, if spotted, can lead to a checkmate or winning advantage.
  Players often practice these with chess puzzles: positions where a tactic is
  known to exist and must be found. Different types of tactics -- and hence
  puzzles -- exist, all with various difficulties. These are almost always
  human-curated and this makes it difficult to automatically find similar
  positions to practice, especially if the tactic is rare.

  This project explores two distinct novel ways to classify chess puzzles. The
  first is a transformer-based convolutional neural network with the goal of
  predicting pre-defined tactics labels and difficulty from static chess
  positions. The second is a distance function between chess moves which is
  used to compare the edit distance of computer-generated move trees in chess
  puzzles. This is used for unsupervised clustering of chess puzzles, and for
  tactics and difficulty prediction with the $k$-Nearest Neighbour algorithm.

  Both methods are successful at both of these tasks. The two methods
  respectively achieve micro-averaged F1-scores of $0.6689$, $0.6893$ on
  tactics label prediction and $r^2$ scores of $0.4256$, $0.2593$ on difficulty
  rating prediction. The former shows evidence of learning the chess piece
  moveset, suggesting that transformer methods can find success in the world of
  chess. The latter, along with allowing positions to be compared by distance,
  shows that standard clustering methods can be used to create meaningful
  groups of puzzles.

\end{abstract}

\renewcommand{\abstractname}{Acknowledgements}
\begin{abstract}

  I would like to thank my supervisor, Nicolas Wu, for his guidance, time, and
  laissez-faire approach, which helped me thrive and enjoy this project.

  I express my gratitude to my friends and family who helped and supported
  me during this project, and the rest of my studies. 

\end{abstract}

\renewcommand{\abstractname}{Ethical Discussion}
\begin{abstract}

  This project has few, if any, ethical considerations that need to be made.
  The biggest concern is the source of the large datasets that may be used in
  this project, namely, the Lichess puzzle database. However, this database,
  along with all Lichess games is released under the Creative Commons CC0
  license -- meaning there is no restriction in its use.

  Each puzzle in this database contains a link to the game from which it was
  extracted, and this contains a Lichess username, which is deanonymising. In
  the initial data processing stage, these links are going to be removed, as
  they are unnecessary for the goals of the project.

  Another concern is the use of, and reference to specific chess games and
  the players in them, and various chess compositions\footnote{An
  artificially created position to showcase a uniquely challenging, rare, or
  otherwise interesting theme.} with their authors. Other than proper
  reference to the creators of the game or composition, there are no other
  considerations to be made.

\end{abstract}

\tableofcontents
%\listoffigures
%\listoftables

\chapter{Introduction}

Chess has long been analysed and enjoyed by both amateurs and professionals
alike, and with the birth of computer science, an algorithm, along with a
powerful enough computer to run it on, has been hunted for since at least the
1950s \citep{shannon}. In 1997, Garry Kasparov, the strongest chess Grandmaster
at the time, was defeated by Deep Blue \citep{deepBlue}, and in the world of
chess, humans were displaced by machines forever. 

Despite this, chess remains a popular pastime for many people, with the largest
online chess website, \url{chess.com}, having over 160,000,000 total members
\citep{chesscomMembers}. Whilst chess engines compete at the superhuman level
with cutting-edge techniques, many average chess players refine their ability
with study of previous positions -- chess puzzles -- where the goal is to find
the critical move to punish the opponent's mistake and gain a decisive
advantage, or win the game immediately via checkmate. 

These puzzles seek to train the `tactics' element of one's chess ability, and
they can be categorised by the theme that occurs within them; some of these
include \emph{fork}ing enemy pieces with a knight, a well-known bishop
sacrifice known as the \emph{Greek gift sacrifice}, and many more, in addition
to their combinations \citep{chessPatterns}. This allows players to identify
which types of patterns they are often overlooking in real games, and to
selectively train to detect those. 

It is surprising that this categorisation is most often done manually, be it
via community voting \citep{lichessPuzzles}, or a list of rules that the puzzle
must satisfy to be classed one way or another \citep{lichessTagger}. This,
along with Teichmann's infamous saying `chess is 99\% tactics', suggests that
there is a need to explore this problem and provide in-depth categorisation of
puzzles (beyond what is currently the norm).

Chess Grandmasters often talk about the `feeling' of a position -- is it
possible to quantify this?

\section{Objectives}

The goal of this project is to explore novel ways to design and implement chess
puzzle classifiers. It should improve upon existing solutions by

\begin{enumerate}
    \item Better analysing the tactical themes which occur in the puzzle.
    \item Creating a way to calculate a `distance' between chess puzzles to 
    group puzzles by theme and difficulty similarity.
    \item Calculating the overall difficulty of a problem.
\end{enumerate}

\section{Contributions}

(This section is empty ... for now)

\section{Report Structure}

In Chapter \ref{backgroundChapter}, we cover some of the background of chess
programming, including notation and programmatic representation. Some
pre-existing work is covered, including the Lichess puzzle database.

Chapter \ref{mlChapter} views the problem in the context of deep learning. By
treating individual chess squares as tokens and applying the infamous
transformer, a model to predict puzzle labels is designed and trained.

Chapter \ref{treeChapter} takes a different route. With the help of search
trees, we attempt to capture how a human might compare puzzles. This leads to a
notion of distance between puzzles and lends itself to unsupervised clustering.

Chapter \ref{evalChapter} compares, contrasts, and evaluates both of these
methods.

Finally, Chapter \ref{concChapter} brings everything together and highlights
some related work to this project, to which this project is compared.


\chapter{Background}
The motivation behind solving chess puzzles, especially under time pressure,
is to improve one's pattern recognition abilities. It has been shown \cite{thoughtAndChoice} 
that chess players do not have a square-by-square, recollection of the chess board 
during play. Instead, they rely on interactions, and potential interactions,
of pieces in a more abstract sense. This is made strikingly clear by a chess 
player's drawing of a position from memory (Figure \ref{deGrootFigure}).
\\~\\
de Groot writes, ``the pieces themselves do not appear in the drawing: 
rather the lines of force that go out from them''. This heuristic allows expert
human players to quickly hone in on the correct moves \cite{bilalic2010mechanisms},
and significantly reduces their search space, when compared to a naive brute-force
search, which is the most common strategy for chess engines. When analysing 
positions, even the most basic chess engines are able to calculate the most 
accurate line (except some pathological cases), but they are unable to draw
similarities between different positions.

\begin{figure}[H]
    \centering
    \includegraphics[width=0.9\linewidth]{background/img/deGroot.png}
    \caption{Taken from de Groot's `Thought And Choice In Chess' \cite{thoughtAndChoice}.}
    \label{deGrootFigure}
\end{figure}
\\~\\
In Figures \ref{chess1}, \ref{chess2}, 2 chess positions are shown, featuring 
\emph{back rank checkmates} -- one of the first tactical patterns that a beginner 
might learn. A castled king is forced to stay on the back rank due to the 
configuration of his men, and an opposing heavy piece exploits this weakness
by delivering checkmate. It is already non-trivial to programmatically identify
this type of tactic, as a checkmate delivered to a king on the back rank is not
necessarily a \emph{back rank checkmate}.
\\~\\
Given positions similar to ones like these, how does one draw similarities 
between the abstract relations of the heavy piece, king, and vulnerable back rank?

\begin{figure}[H]
    \begin{minipage}{0.475\textwidth}
        \centering
        \chessboard[setfen=6k1/5ppp/8/8/8/8/r4PPP/1R4K1 w - - 0 1]
        \caption{A trivial backrank checkmate, white mates with \textbf{Rb8\#}
        (TODO: Find real chess games where these positions/similar ones happened)}
        \label{chess1}
    \end{minipage}
    \hspace{0.05\textwidth}
    \begin{minipage}{0.475\textwidth}
        \centering
        \chessboard[setfen=6k1/5ppp/1p1Q4/p3p1B1/Pn4P1/1q6/1Pr4P/K6R w - - 1 2]
        \caption{White mates with \textbf{Qd8\#}.}
        \label{chess2}
    \end{minipage}
\end{figure}

\section{Chess notation}
TODO: Explain chess FEN, PGN notation. The reader is assumed to know how to play 
chess at a basic level
\\~\\
TODO: Show examples of basic puzzles/tactics in chess
\\~\\
TODO: Explain basic chessprogramming, e.g. bitboards, 0x88

\section{A language for encoding piece relationships}
In ``A description language for chess'' \cite{chessLanguage}, López-Michelone et al. 
describe a language to search across chess positions. The main features of this
language are descriptions for a chess piece attacking/defending another, 
attacking/defending a square, being located at a square, a square not being 
available for the enemy king, and the structure of white/black pawns on the board.
\\~\\
With their novel language, they are able to search a chess database for a 
pre-determined pattern, such as the \emph{Greek gift sacrifice}, defined in the language as 
``kg8, pf7, pg7, B(ph7), Nf3, Qd1, Pe5''
\footnote{This string corresponds to a black king on g8, black pawns on f7, g7, h7,
a white bishop attacking h7, a white knight on f3 (ready to deliver a check with \textbf{Ng5+},
a common motif in \emph{Greek gift sacrifices}), a white queen on d1, and finally, a
white pawn on e5 to dislodge the usual black knight on f6.}
, which returns positions such as 
the one shown in Figure \ref{chess3}. After \textbf{16...Be7 17. O-O}, Black blundered with
\textbf{17...Bxa3??}, after which, the \emph{Greek gift sacrifice} (\textbf{18. Bxh7!!}, 
shown in Figure \ref{chess4}) was employed, eventually culminating in a win for White.

\begin{figure}[H]
    \begin{minipage}{0.475\textwidth}
        \centering
        \chessboard[setfen=r1b2rk1/qp3ppp/p1n1pb2/4P3/3P4/P1BB1N2/5PPP/1R1QK2R b K - 0 16]
        \caption{\textbf{Pirc, V -- Porreca, G}, YUG-ITA m 1953, move 16.}
        \label{chess3}
    \end{minipage}
    \hspace{0.05\textwidth}
    \begin{minipage}{0.475\textwidth}
        \centering
        \chessboard[setfen=r1b2rk1/qp3ppB/p1n1p3/4P3/3P4/b1B2N2/5PPP/1R1Q1RK1 b - - 0 18]
        \caption{\textbf{Pirc, V -- Porreca, G}, YUG-ITA m 1953, move 18. Black resigned after 6 moves.}
        \label{chess4}
    \end{minipage}
\end{figure}
\\~\\
Their language is also able to deal with some light variations, as it is able
to identify the games shown in Figures \ref{chess5}, \ref{chess6}. In both of 
these positions, White has the brilliant move \textbf{1. Qh6+!!}, following with 
\textbf{2. Rh8\#} if \textbf{1...Kxh6}, and either \textbf{2. Rf7\#} or \textbf{2. Rb7+}
(leading to a quick mate) if \textbf{1...gxh6}. 
\\~\\
This pattern, whilst very rare, is undeniably identical between the 2 games. 
The unavailability of the \textbf{g6} square to the enemy king, combined with the
harmony of White's pieces leads to the same tactic in both games.

\begin{figure}[H]
    \begin{minipage}{0.475\textwidth}
        \centering
        \chessboard[setfen=2R5/4bppk/1p1p4/5R1P/4PQ2/5P2/r4q1P/7K w - - 5 50]
        \caption{\textbf{Carlsen, M -- Karjakin S}, World Chess Championship 2016, move 50.}
        \label{chess5}
    \end{minipage}
    \hspace{0.05\textwidth}
    \begin{minipage}{0.475\textwidth}
        \centering
        \chessboard[setfen=5R2/bp4pk/2n3p1/P7/P1q3bP/6P1/3Q3K/1R6 w - - 1 32]
        \caption{\textbf{Popov, N -- Novopashin, A}, URS-ch otbor 1979, move 32.}
        \label{chess6}
    \end{minipage}
\end{figure}
\\~\\
The work of López-Michelone et al. is a promising proof of concept that shows
the power of a language that allows to specify piece relationships on a more
abstract level than previously possible. The biggest drawback of their solution,
as mentioned by the authors, is the fact that this language still requires 
an expert with pre-existing extensive knowledge to encode the tactics into
their language. The authors hypothesise that automatic recognition of these 
patterns is likely some sort of neural network, which is one of the many possible
directions of this project.


\section{CQL: Chess Query Language}
The Chess Query Language (CQL), as invented by Costeff \cite{cql}, is another 
implementation of an advanced way to find chess positions in a given database.
Since its inception in 2004, it has grown and is able to support very powerful,
sometimes esoteric, queries to find predefined patterns.
\\~\\
TODO: Show examples from \url{http://www.gadycosteff.com/cql/examples/smotheredmate.html}
and \url{http://www.gadycosteff.com/cql/examples/turton.html}
\\~\\
In addition to the Costeff's CQL, there is a from-scratch clone of CQL 6.1 which includes
extra features (TODO: list some) and supports other chess variants \cite{cqli}.

\section{lichess.org puzzle database}
\url{https://lichess.org} is a popular, open-source chess website which often
publishes the games that have been played by players of all skill levels on it.
As part of this, Lichess has published over 3.6 million rated and tagged
puzzle positions \cite{lichessPuzzles}. To generate these, 300 million games
were analysed with a powerful chess engine to find critical positions in which
a move must be played to capitalise on the opponent's mistake. These puzzles
were initially tagged to 124 manually created themes \cite{lichessXML}, which
were identified by a python implementation \cite{lichessTagger}. 
\\~\\
As various users of the site solved the puzzles, and manually highlighted the
themes that they felt occured in the puzzle, the ratings and tags of the puzzle
database evolved until their current state.
\\~\\
This database is invaluable for this project, and can likely serve either as input,
or as a baseline to compare the results to.

\section{Recognition of the \emph{Greek gift sacrifice} in  chess games}
Miroslav et al. report their findings on a program \cite{middlegamePatterns}
to identify the \emph{Greek gift sacrifice} in chess middlegames
\footnote{A hard-to-quantify phase of the chess game where
most pieces are developed and the kings are positioned away from danger.}. They
found success, partly thanks to their detailed representatioon of the board, where
each square is represented by 71 binary attribites \cite{middlegamePatterns} to
encode the piece on the square and the possible squares which are reachable by
this piece. 
\\~\\
These attributes were also supplemented by 59 other binary attribites which were
devised by expert chess players, and represent the more complex, but still
quantifiable relationships \cite{middlegamePatterns}. These include open files,
control of vulnerable squares, piece activity, and so on.
\\~\\
Miroslav et al. achieved an 87.7\% classification performance on detecting whether
a position features a successful \emph{Greek gift sacrifice} on relatively 
small (<200 positions) datasets. This work is promising, but more investigation
is needed given their analysis of only one pattern with a decision tree algorithm.
Also, their program still relied on predetermined chess patterns, which introduces
bias and might miss intricate similarities between positions.

\section{A novel chess board representation for convolutional neural networks}
In Sabatelli's thesis \cite{chessCNN}, the effectiveness of neural networks to analyse whether
a chess position is winning or losing is explored, without creating specific
look-ahead algorithms. This is a very challenging task, but as part of the analysis,
the author proposed manually encoded features to a convolutional neural network (CNN)
to help identify strong patterns within the position. 
\\~\\
Some of these include an extra feature if the opponent is in check, specifying
the squares controlled by a piece exerting a pin on a different piece, centre
control, and vulnerable squares \cite{chessCNN}. These are well known heuristics that are often
taught to beginner-intermediate chess players, and the author claims that these
additional layers are ``extremely representative of the chess positions'', and
cause the CNN to outperform a naive, fully-connected neural network.
\\~\\
This work is also a promising result, as it shows that these chess patterns,
albeit manually quantified, are valuable for an algorithm when analysing a given
chess position.

\section{Chess moves as kernels for texture classifier}
In this study, Turker et al. propose novel kernels for efficient feature extraction
in the task of texture detection \cite{chessKernel}. These 5x5 kernels are
directly based on the move a rook, bishop, knight, and their combinations. 
\\~\\
Whilst not directly applicable to the context of chess puzzle analysis, this work
shows that it may be possible to include these kernels in an CNN-based analysis
of the chess position. It would be interesting to apply to the other CNN chess work
(e.g. Sabatelli's thesis \cite{chessCNN}, discussed above) and analyse its
effect on the success of the technique.

\section{TODO: Research + summarise more papers}

\section{Ethical discussion}
This project has few, if any, ethical considerations that need to be made. The
biggest concern is the source of the large datasets that may be used in this 
project, namely, the Lichess puzzle database \cite{lichessPuzzles}. However,
this database, along with all Lichess games is released under the 
Creative Commons CC0 license -- meaning there is no restriction in its use.
\\~\\
Each puzzle in this database contains a link to the game from which it was
extracted, and this contains a Lichess username, which is deanonymising. In
the initial data processing stage, these links are going to be removed, as
they are unnecessary for the goals of the project.
\\~\\
Another concern is the use of, and reference to specific chess games and the
players in them, and various chess compositions\footnote{An artificially 
created position to showcase a uniquely challenging, rare, or otherwise interesting
theme.} with their authors. Other than proper reference to the creators of the game or
composition, there are no other considerations to be made.
\chapter{Implementation}

\section{Lichess.org Puzzle Database}

\subsection{Overview}

This project requires access to various examples of chess puzzles with
pre-defined difficulties and themes. Fortunately, the lichess puzzle
database,\cite{lichessPuzzles} which was also mentioned in the previous
section, provides approximately 3.8 million chess puzzles which have been
generated from user games played on lichess. These puzzles are stored in FEN
format, with a reference to the game where they appeared. They also contain the
solution as a string moves, the tactics tags,\cite{lichessXML} and the puzzle
rating, along with other metadata.

An example puzzle string is shown below. This puzzle is also shown in the two
figures below, Figure \ref{puzzle1} and \ref{puzzle2}.

\begin{verbatim}
q3k1nr/1pp1nQpp/3p4/1P2p3/4P3/B1PP1b2/B5PP/5K2 b k - 0 17,
e8d7 a2e6 d7d8 f7f8,1760,80,83,72,mate mateIn2 middlegame short,
https://lichess.org/yyznGmXs/black#34,
Italian_Game Italian_Game_Classical_Variation
\end{verbatim}

\begin{figure}[H]
    \begin{minipage}{0.475\textwidth}
        \centering
        \chessboard[setfen=q3k1nr/1pp1nQpp/3p4/1P2p3/4P3/B1PP1b2/B5PP/5K2 b k - 0 17]
        \caption{\textbf{ZensAlviani -- desso2b}, lichess.org Blitz game, move 17.}
        \label{puzzle1}
    \end{minipage}
    \hspace{0.05\textwidth}
    \begin{minipage}{0.475\textwidth}
        \centering
        \chessboard[setfen=q5nr/1ppknQpp/3p4/1P2p3/4P3/B1PP1b2/B5PP/5K2 w - - 1 18]
        \caption{\textbf{ZensAlviani -- desso2b}, lichess.org Blitz game, move 18.
        Checkmate is imminent with \texttt{18.Be6+ Kd8 19.Qf8\#}.}
        \label{puzzle2}
    \end{minipage}
\end{figure}

Processing these is a trivial task with one small detail: the given FEN strings
are the state of the game right before the critical blunder is played by the
opposing site. This means the puzzle, as shown to the user, is the position
after the first move has been played. Fortunately, processing this data is made
simple with the python-chess library.\cite{pythonChess}

\subsection{Data Exploration}

The lichess puzzle database has approximately 3.8 million rated and tagged
chess puzzles. Initially, they were automatically
processed,\cite{lichessTagger} but were then refined with user
feedback.\cite{lichessPuzzles} This also allowed the puzzles to obtain a
rating, which is indicative of its difficulty.\cite{lichessPuzzles}

Overall, there are 60 various puzzle themes.\cite{lichessXML} Figure
\ref{dataThemeCounts} shows the counts of each theme in all of the contained
puzzles. It should be noted that some themes are mutually exclusive (a
checkmate puzzle cannot be both \emph{mate-in-one} and \emph{mate-in-two}).

The most common puzzle themes are the most general ones -- specifically
`short', `middlegame', and `endgame'. There is a lot of variation in how
frequent the various patterns are, which is a natural consequence of the game.

\begin{figure}
    \centering
    \includegraphics[width=0.9\linewidth]{project/img/puzzle_theme_counts.png}
    \caption{Frequency of puzzle themes in the lichess.org puzzle database.\cite{lichessPuzzles}}
    \label{dataThemeCounts}
\end{figure}

Figure \ref{dataHistogram}, shown below, shows the distribution of ratings
across the chess puzzles. It should be noted that these ratings are only
appropriate within this dataset, and cannot be compared to ratings of puzzles
on other chess websites. The puzzle ratings are quite symmetric about the mean,
and this is of course a result of Glickman's Glicko2 rating
system.\cite{glicko}

When analysing the rating distribution by theme, an expected behaviour occurs.
Some chess tactics patterns are considerably simpler to spot, meaning a weaker
player is able to solve the puzzle with that theme. Therefore some puzzle
themes, \emph{back-rank mate}, for example, have lower-rated puzzles when
compared to puzzles featuring a \emph{trapped piece} or \emph{defensive
move}.\footnote{Notoriously difficult for humans, who are usually much better at
aggression than defense.}

Shown in Figures \ref{puzzle3} and \ref{puzzle4} are examples of some puzzles
with these themes.

\begin{figure}[H]
    \begin{minipage}{0.475\textwidth}
        \centering
        \includegraphics[width=\textwidth]{project/img/puzzle_histogram.png}
        \caption{Distribution of lichess.org puzzle ratings.}
        \label{dataHistogram}
    \end{minipage}
    \hspace{0.05\textwidth}
    \begin{minipage}{0.475\textwidth}
        \centering
        \includegraphics[width=\textwidth]{project/img/puzzle_theme_histograms.png}
        \caption{Distribution of lichess.org puzzle ratings with a specific theme.}
        \label{dataThemeHistogram}
    \end{minipage}
\end{figure}

\begin{figure}[H]
    \begin{minipage}{0.475\textwidth}
        \centering
        \chessboard[setfen=6k1/5ppp/r1p5/p1n1rP2/8/2P2N1P/2P3P1/3R2K1 w - - 0 22]
        \caption{\textbf{Kenan2345 -- gandie}, lichess.org Blitz game, move 22. 
        Black loses to \texttt{22.Rd8+}.}
        \label{puzzle3}
    \end{minipage}
    \hspace{0.05\textwidth}
    \begin{minipage}{0.475\textwidth}
        \centering
        \chessboard[setfen=2rq1rk1/7p/1n4pb/1R2Q3/pPpP1P2/P1B5/3N2PP/2R3K1 b - - 0 31]
        \caption{\textbf{mhabib -- Sarg0n}, lichess.org Blitz game, move 31. White loses his queen after \texttt{31...Re38}, as the queen has no safe squares to escape to.}

        \label{puzzle4}
    \end{minipage}
\end{figure}

\section{The Machine Learning Approach}

\subsection{Introduction}

In this section, we describe, analyse, and evaluate a novel approach to the
specific problem of chess puzzle classification, inspired by the recent
unstoppable advancements in the field of natural language processing.

Earlier, we highlighted a number of papers which seek to find new ways to build
on the naive bitboard
representation,\cite{middlegamePatterns}\cite{chessCNN}\cite{chess2vec} by
exploring new embeddings for chess pieces and chess board squares. All three of
the publications make the crucial point that chess pieces influence each other
on the board, and this has to be taken into account, whether it is by creating
extra features to represent pins and central square control,\cite{chessCNN}
open files and attack squares,\cite{middlegamePatterns} or the hash of the
entire chess board.\cite{chess2vec}

Continuing along the `chess board as a $N\times64$ vector' path and, given how
puzzle tactics rely on the interaction of pieces' locations and attack squares,
it seems natural to explore whether the transformer architecture, as described
in the infamous paper `Attention is All you Need',\cite{attention} can help
extract patterns from static chess positions, which can then be used in the
downstream task of puzzle classification and difficulty rating. Figure
\ref{attentionLinks}, taken from this paper, shows how words/tokens can
influence each other in the transformer encoder. At a high level, chess pieces
and squares interact in a similar way on the chess board -- see Figure
\ref{chessPuzzleLinks} for an example.

Applying transformers to chess has been done before,\cite{chessTransformer} but
as far as we are aware, treating individual chess board squares as embeddings
of tokens/words has not been previously explored.

\begin{figure}[H]
    \begin{minipage}{0.425\textwidth}
        \centering
        \includegraphics[width=\textwidth]{project/img/attention.png}
        \caption{Example of attention mechanism in the encoder. Taken from `Attention Is All You Need'.\cite{attention}}
        \label{attentionLinks}
    \end{minipage}
   \hspace{0.05\textwidth}
    \begin{minipage}{0.475\textwidth}
        \centering
        \chessboard[setfen=r3r1n1/bp6/p2p2kp/3N4/2P3n1/1PQ3Pq/P4P2/4RRK1 w - - 0 1,
                    pgfstyle=border,markfields={d5},
                    pgfstyle=straightmove,markmoves={f4-g6,f4-h3,c7-a8,c7-e8,d5-c7,d5-f4},
                    pgfstyle=color,opacity=0.5,color=blue,markfields={f4,c7},
                    pgfstyle=color,opacity=0.5,color=red,markfields={h3,g6,a8,e8}]
        \caption{Example of a relationship between chess pieces.}

        \label{chessPuzzleLinks}
    \end{minipage}
\end{figure}

\subsection{Data Preparation}

We formulate the problem of chess puzzle analysis as a multi-label
classification problem. Given a chess puzzle position in bitboard format,
i.e.\@ a $12\times8\times8$ binary tensor, we aim to predict a binary indicator
vector which encodes the labels of the puzzle themes. Slightly more formally,
the element $B[p, r, f]$ of bitboard $B$ is $1$ iff there is a piece $z \in
\{p,P,n,N,b,B,r,R,q,Q,k,K\}$\footnote{FEN-inspired -- white pieces are
uppercase, black pieces are lowercase.} at rank $r \in \{1,2,3,4,5,6,7,8\}$,
file $f \in \{a,b,c,d,e,f,g,h\}$.

Given a dataset with $N$ distinct puzzle labels ($N=60$ for the Lichess puzzle
database \cite{lichessPuzzles}), we assign each label an index $i$ and form a
binary $N$-dimensional vector $x$ such that $x[i]=1$ iff the puzzle is labelled
with that label. This is a common technique to reduce a multi-label
classification into a more tractable problem.

To make learning easier (and to decrease implementation complexity), we
transform all chess positions to be from the perspective of White. This has
negligible additional overhead, as it requires a simple mirror and has great
benefits, as the model no longer needs to distinguish between white-to-move and
black-to-move puzzles.

\subsection{Model Architecture}

\begin{figure}[H]
        \centering
        \includegraphics[width=\textwidth]{project/img/ml_diagram.png}
        \caption{High-level overview of the model architecture. Transformer encoder diagram taken from `Attention Is All You Need'.\cite{attention}}
        \label{MLDiagram}
\end{figure}


\section{The Graph-Based Approach}

\subsection{Introduction}

In this section, we propose a different novel approach to puzzle
classification, focusing on unsupervised puzzle clustering and defining a
distance function between various chess puzzles.

This method seeks to combine and build upon two approaches seen in the above
literature review: the tree-based puzzle difficulty classification
\cite{chessTrees} covered in Section \ref{chessTreesOverview} and chess
position similarity using dynamic features \cite{chessMotifs} covered in
Section \ref{chessMotifsOverview}. We hypothesise that by combining these
approaches to construct meaningful search trees with additional node labels,
defining a distance function for individual chess moves, and applying a
labelled tree edit distance function,\cite{editDistTrees} we would be able to
calculate `closeness' of chess puzzles. This could also help refine the work of
Stoiljkovikj et al.\@'s by finer segregating puzzles by difficulty level.

\subsection{High-level overview}

As explored by Stoiljkovikj et al.\@, `meaningful search trees' have predictive
possibility of a puzzle's difficulty.\cite{chessTrees} These trees are
constructed by analysing powerful moves that either gain, or at least do not
worsen a side's position. 

By additionally labelling these trees with move information, we hope to build
upon this work. To demonstrate the intuition behind this idea, there are two
labelled search trees shown below in Figures \ref{tree1}, \ref{tree2}. These
are the search trees for positions in Figures \ref{chess5}, \ref{chess6} which
feature a identical and complex tactical motif. 

In the following trees, algebratic notation of the moves is shown, along with 4
slash-delimited integers, which correspond to: pieces attacked, pieces
defended, number of attackers, number of defenders.\footnote{As the moves must
be legal, this means a king move will never have any attackers, and any
check/mate will have at least 1 piece attacked -- the enemy king.} Whilst the 2
trees do not look similar visually (their size being the main problem), the
first 3 plies are very similar. 

\begin{figure}[H]
    \begin{minipage}{0.475\textwidth}
        \centering
        \includegraphics[width=\textwidth]{project/img/trees/1.drawio.png}
        \caption{Labelled search tree for game in \ref{chess5}}
        \label{tree1}
    \end{minipage}
    \hspace{0.05\textwidth}
    \begin{minipage}{0.475\textwidth}
        \centering
        \includegraphics[width=\textwidth]{project/img/trees/2.drawio.png}
        \caption{Labelled search tree for game in \ref{chess6}}
        \label{tree2}
    \end{minipage}
\end{figure}

Below, in Figures \ref{puzzle5}, \ref{puzzle6} are two chess puzzles from the
paper `Automatic Recognition of Similar Chess Motifs'.\cite{chessMotifs}
Despite their visual difference, these both feature a rook sacrifice, and an
imminent queen checkmate helped by the powerful light-squared bishop. This
similarity is not immediately obvious, but the search trees for these puzzles
is almost identical, except for minor attacker/defender discrepancies. The
trees are shown in Figures \ref{tree3}, \ref{tree4}. These puzzles were already
successfully grouped by Bizjak et al.\@.\cite{chessMotifs}

\begin{figure}[H]
    \begin{minipage}{0.475\textwidth}
        \centering
        \chessboard[setfen=4r1k1/1b3pp1/4p3/p2r4/7R/2B1Q1PP/P1P1RP1K/1q6 w - - 0 1]
        \caption{Taken from `Automatic Recognition of Similar Chess Motifs'.\cite{chessMotifs} White mates in 3 (\texttt{1.Rh8+ Kxh8 2.Qh6+ Kg8 3.Qxg7#}).}
        \label{puzzle5}
    \end{minipage}
    \hspace{0.05\textwidth}
    \begin{minipage}{0.475\textwidth}
        \centering
        \chessboard[setfen=r5k1/5pp1/8/3p3R/2q4P/PbB2P2/1P1Q2P1/K7 w q - 0 1]
            \caption{Also taken from `Automatic Recognition of Similar Chess Motifs'.\cite{chessMotifs} White mates in 3 with the same moves as Figure \ref{puzzle5}.}
        \label{puzzle6}
    \end{minipage}
\end{figure}

\begin{figure}[H]
    \begin{minipage}{0.475\textwidth}
        \centering
        \includegraphics[width=\textwidth]{project/img/trees/3.drawio.png}
        \caption{Labelled search tree for game in \ref{puzzle5}}
        \label{tree3}
    \end{minipage}
    \hspace{0.05\textwidth}
    \begin{minipage}{0.475\textwidth}
        \centering
        \includegraphics[width=\textwidth]{project/img/trees/4.drawio.png}
        \caption{Labelled search tree for game in \ref{puzzle6}}
        \label{tree4}
    \end{minipage}
\end{figure}


\chapter{Evaluation}\label{evalChapter}

After explaining the two approaches, they are now compared on the tasks
outlined at the beginning (\Cref{introductionChapter}): how good are they at
predicting the tactical themes in a puzzle (\Cref{evalS1}), and how good are
they at predicting the difficulty of a puzzle (\Cref{evalS2})?

Each of the methods is evaluated in greater detail (\Cref{evalS3,evalS4}),
focusing on their individual advantages and disadvantages
(\Cref{evalS31,evalS41}).

Finally, each method has a unique application that the other does not: a game
review for the deep learning transformer approach (\Cref{evalS32}) and a
similarity comparison for the tree-based approach (\Cref{evalS42}). These are
discussed and judged based on their own merits.


\section{Prediction of Puzzle Themes}\label{evalS1}

Both approaches (\Cref{mlChapter,treeChapter}) are compared with the unseen
test set. The deep learning approach produced a model which can be tested
against the entire test set with no further modifications, while the $k$-NN
method (\Cref{treeS3}) is used to evaluate the tree-based approach. Due to its
computational complexity (\Cref{treeS31}), $k$-NN uses a subset of 100,000
puzzles of the training set and is evaluated against a random selection of
2,000 puzzles from the test set. This, of course, increases variance in the
final evaluation of the model, but is unavoidable without great investments in
time and processing power.

Per-label precision, recall, and F1-scores are considered for both approaches
(\Cref{labelTable}). Additionally, the averaged statistics are also shown. In
all averaging formats (micro, macro, weighted, by-sample), the deep learning
transformer model outperforms the tree-based $k$-NN approach, but it is
surprisingly close. The macro average (unweighted mean of the statistics)
suffers greatly for both approaches. This is due to the heavily unbalanced
dataset (\Cref{lichessDataExpl}). The most appropriate metrics to compare are
the weighted averages, where more frequent themes have a higher weight, and
samples average, where the scores are calculated for each individual sample.
The latter is most useful for multilabel classfication.

The theme breakdown, despite being difficult to parse, shows some interesting
differences between the models' performances. All themes discussed in this
section are highlighted in \Cref{labelTable} for ease of readability. The
tree-based algorithm, which uses Stockfish for tree generation
(\Cref{treeS12}), is perfect at identifying \texttt{mateIn1} and almost perfect
at \texttt{mateIn2}. This is likely because mating patterns are quite common
and the meaningful move trees constructed for them are almost identical. For
this same reason, the tree-based model outperforms the deep learning
transformer model at \texttt{oneMove} and \texttt{short} label prediction.

On the other hand, the deep learning transformer model is almost completely
accurate at predicting \texttt{middlegame} and \texttt{endgame} labels, along
with the various types of endgame: \texttt{bishopEndgame},
\texttt{knightEndgame}, \texttt{pawnEndgame}, \texttt{queenEndgame},
\texttt{queenRookEndgame}, and \texttt{rookEndgame}. This is almost certainly a
result of the model being designed in a way where it can access the whole board
at once, and an absence of many pieces (which is a giveaway that a position is
in an endgame), is trivial for it to detect. 


\begin{table}[H]
  \centering
  \begin{adjustbox}{width=\textwidth}
    \begin{tabular}{l|rrrr|rrrr}
      & \multicolumn{4}{c}{Tree-Based $k$-NN} & \multicolumn{4}{c}{Deep Learning Transformer} \\
      Theme & Precision & Recall & F1-score & Support & Precision & Recall & F1-score & Support \\
      \hline

      advancedPawn&0.7660&0.3600&0.4898&100&0.7044&0.4462&0.5463&43215\\
      advantage&0.5693&0.4091&0.4761&572&0.5966&0.5898&0.5932&235529\\
      anastasiaMate&1.0000&0.3333&0.5000&3&0.8554&0.5338&0.6574&798\\
      arabianMate&0.0000&0.0000&0.0000&1&0.7858&0.5842&0.6702&760\\
      attackingF2F7&0.8333&0.5556&0.6667&9&0.8693&0.7412&0.8002&4657\\
      attraction&0.9697&0.4156&0.5818&77&0.6858&0.1767&0.2810&26566\\
      backRankMate&0.6607&0.5692&0.6116&65&0.8236&0.8427&0.8330&24332\\
      \rowcolor{lightgray} bishopEndgame&0.7778&0.4667&0.5833&15&0.9622&0.9774&0.9697&9476\\
      bodenMate&0.0000&0.0000&0.0000&2&0.8045&0.5619&0.6617&315\\
      capturingDefender&0.0000&0.0000&0.0000&13&0.7120&0.0209&0.0406&6505\\
      castling&0.0000&0.0000&0.0000&2&0.5000&0.0071&0.0141&421\\
      clearance&0.5000&0.0400&0.0741&25&0.7768&0.0320&0.0616&10641\\
      crushing&0.7094&0.6704&0.6893&892&0.7223&0.5972&0.6538&337162\\
      defensiveMove&0.5000&0.1121&0.1832&107&0.5340&0.0462&0.0850&47212\\
      deflection&0.7857&0.1341&0.2292&82&0.7243&0.0940&0.1664&32867\\
      discoveredAttack&0.7778&0.2823&0.4142&124&0.6760&0.2641&0.3799&43352\\
      doubleBishopMate&0.0000&0.0000&0.0000&0&0.7907&0.5795&0.6689&352\\
      doubleCheck&0.0000&0.0000&0.0000&8&0.7518&0.2327&0.3554&3670\\
      dovetailMate&0.0000&0.0000&0.0000&2&0.7222&0.0323&0.0619&402\\
      enPassant&0.0000&0.0000&0.0000&0&0.7333&0.0189&0.0368&1165\\
      \rowcolor{lightgray} endgame&0.8036&0.7412&0.7711&966&0.9984&0.9999&0.9992&368846\\
      equality&0.0000&0.0000&0.0000&23&1.0000&0.0001&0.0002&8503\\
      exposedKing&0.0000&0.0000&0.0000&54&0.3561&0.1114&0.1697&21239\\
      fork&0.7500&0.5445&0.6309&281&0.7666&0.4015&0.5270&111331\\
      hangingPiece&0.6538&0.2073&0.3148&82&0.6806&0.1650&0.2657&33331\\
      hookMate&0.0000&0.0000&0.0000&5&0.6887&0.4616&0.5527&1107\\
      interference&0.0000&0.0000&0.0000&9&0.6250&0.0016&0.0032&3129\\
      intermezzo&1.0000&0.0357&0.0690&28&0.7517&0.0402&0.0763&11221\\
      kingsideAttack&0.7742&0.3357&0.4683&143&0.7146&0.6082&0.6571&61522\\
      \rowcolor{lightgray} knightEndgame&0.7000&0.4667&0.5600&15&0.9319&0.9944&0.9621&5887\\
      long&0.6809&0.1932&0.3009&497&0.5746&0.0754&0.1333&193731\\
      master&0.0000&0.0000&0.0000&148&0.2500&0.0001&0.0001&54003\\
      masterVsMaster&0.0000&0.0000&0.0000&11&0.0000&0.0000&0.0000&4967\\
      mate&0.9783&0.9688&0.9735&512&0.8210&0.7731&0.7963&198507\\
      \rowcolor{lightgray} mateIn1&1.0000&1.0000&1.0000&219&0.7483&0.6803&0.7127&80324\\
      \rowcolor{lightgray} mateIn2&0.9773&1.0000&0.9885&215&0.7727&0.5100&0.6145&91787\\
      mateIn3&0.9388&0.6866&0.7931&67&0.7096&0.2987&0.4204&22611\\
      mateIn4&0.6667&0.2500&0.3636&8&0.5562&0.0322&0.0608&3076\\
      mateIn5&0.0000&0.0000&0.0000&3&0.9062&0.0409&0.0783&709\\
      \rowcolor{lightgray} middlegame&0.7285&0.7293&0.7289&931&0.9919&0.9854&0.9886&377218\\
      \rowcolor{lightgray} oneMove&1.0000&0.8939&0.9440&245&0.7566&0.6041&0.6718&89681\\
      opening&0.5909&0.1262&0.2080&103&0.8873&0.9419&0.9138&44379\\
      \rowcolor{lightgray} pawnEndgame&0.7846&0.9623&0.8644&53&0.9742&0.9998&0.9868&22943\\
      pin&0.8261&0.1397&0.2390&136&0.6821&0.2167&0.3288&49568\\
      promotion&1.0000&0.0968&0.1765&31&0.6591&0.3415&0.4499&16317\\
      \rowcolor{lightgray} queenEndgame&1.0000&0.2500&0.4000&16&0.9748&0.9915&0.9830&7282\\
      \rowcolor{lightgray} queenRookEndgame&0.0000&0.0000&0.0000&17&0.9482&0.9170&0.9324&5171\\
      queensideAttack&0.7143&0.1471&0.2439&34&0.5973&0.4695&0.5257&10552\\
      quietMove&0.9231&0.1579&0.2697&76&0.5908&0.1497&0.2388&31019\\
      \rowcolor{lightgray} rookEndgame&0.7400&0.3814&0.5034&97&0.9759&0.9893&0.9826&37058\\
      sacrifice&0.8571&0.3947&0.5405&152&0.7042&0.2853&0.4061&55181\\
      \rowcolor{lightgray} short&0.7341&0.8385&0.7829&1090&0.6810&0.6802&0.6806&434161\\
      skewer&0.6667&0.2927&0.4068&41&0.7135&0.2891&0.4115&17291\\
      smotheredMate&0.5000&0.2500&0.3333&4&0.8920&0.8392&0.8648&2264\\
      superGM&0.0000&0.0000&0.0000&0&0.0000&0.0000&0.0000&462\\
      trappedPiece&1.0000&0.1481&0.2581&27&0.6613&0.1431&0.2352&11135\\
      underPromotion&0.0000&0.0000&0.0000&0&0.0000&0.0000&0.0000&140\\
      veryLong&0.7500&0.0180&0.0351&167&0.6098&0.0084&0.0166&62128\\
      xRayAttack&0.0000&0.0000&0.0000&5&0.7980&0.2401&0.3691&2699\\
      zugzwang&0.4444&0.2000&0.2759&20&0.7079&0.3097&0.4308&6604\\
      \hline
      micro~avg&0.7774&0.5869&0.6689&8630&\textbf{0.8025}&\textbf{0.6040}&\textbf{0.6893}&3388481\\
      macro~avg&0.5339&0.2801&0.3324&8630&\textbf{0.7065}&\textbf{0.3996}&\textbf{0.4497}&3388481\\
      weighted~avg&0.7416&0.5869&0.6211&8630&\textbf{0.7600}&\textbf{0.6040}&\textbf{0.6388}&3388481\\
      samples~avg&0.7501&0.5952&0.6488&8630&\textbf{0.8063}&\textbf{0.6181}&\textbf{0.6734}&3388481\\
    \end{tabular}
  \end{adjustbox}
  \caption{Label breakdown of precision, recall, and F1-score for the two models.}
  \label{labelTable}
\end{table}


\section{Prediction of Puzzle Difficulty}\label{evalS2}

Both methods were also evaluated on their ability to correctly predict
difficulty rating of puzzles. To measure their performance, the average root
mean squared error and $r^2$ score was calculated over the entire test set
(\Cref{diffTable}). The $r^2$ score, the coefficient of determination,
indicates how much of the variance in the results is explined by the model --
higher is better. As before, the tree-based $k$-NN method was only tested on
2,000 samples due to its slow inference time. 

A random sample of 1,000 puzzles, along with their actual and predicted
difficulty ratings, are also shown for each method
(\Cref{predTree,predTransformer}).

From the results, we see that the tree-based $k$-NN is considerably better than
the transformer-based model at predicting difficulty rating, with a root mean
square error of 64 points lower. Additionally, its $r^2$-score is much higher.
It should be noted that this model was evaluated on far fewer samples, but
despite this, its performance is very good.

It is interesting that the transformer-based model tends to overshoot the
rating when it is low, and tends to predict a lower rating when it is high
(\Cref{errTransformer}). This is likely a result of slight overfitting, or a
consequence of the loss function used.

\begin{table}[H]
  \centering
  \begin{tabular}{l|rr}
    Metric & Tree-Based $k$-NN & Deep Learning Transformer \\
    \hline
    $r^2$ score & \textbf{0.4256} & 0.2593 \\
    Root mean squared error & \textbf{404.5} & 468.4 \\
  \end{tabular}
  \caption{Metrics of success for difficulty prediction.}
  \label{diffTable}
\end{table}

\begin{figure}[H]
  \begin{minipage}{0.475\textwidth}
    \centering
    \includegraphics[width=\textwidth]{evaluation/img/tree.png}
    \caption{Actual vs predicted ratings for the tree-based $k$-NN method.}
    \label{predTree}
  \end{minipage}
  \hspace{0.05\textwidth}
  \begin{minipage}{0.475\textwidth}
    \centering
    \includegraphics[width=\textwidth]{evaluation/img/transformer.png}
    \caption{Actual vs predicted ratings for the transformer-based model.}
    \label{predTransformer}
  \end{minipage}
\end{figure}

\begin{figure}[H]
  \begin{minipage}{0.475\textwidth}
    \centering
    \includegraphics[width=\textwidth]{evaluation/img/tree_err.png}
    \caption{Residuals vs predicted ratings for the tree-based $k$-NN method.}
    \label{errTree}
  \end{minipage}
  \hspace{0.05\textwidth}
  \begin{minipage}{0.475\textwidth}
    \centering
    \includegraphics[width=\textwidth]{evaluation/img/transformer_err.png}
    \caption{Residuals vs predicted ratings for the transformer-based model.}
    \label{errTransformer}
  \end{minipage}
\end{figure}


\section{Deep Learning Approach}\label{evalS3}

\subsection{Strengths and Weaknesses}\label{evalS31}

The advantages of this approach are clearly the better performance
(\Cref{labelTable}) compared to the tree-based approach. After a relatively
long training period, the models produced by this method can analyse new
positions almost instantly. As the output of the model is a number in the range
$(0, 1)$, a sense of `confidence' in the output can also be easily understood,
which is considerably more difficult in the tree-based approach with $k$-NN.

The biggest downside is the trade-off of having fast inference: slow training
time. This, combined with the need to search for hyperparameters (\Cref{mlS22})
makes this method quite time and resource intensive. The deep learning method
also requires a lot of access to chess puzzle data, which can be of varying
quality. This also means that it cannot learn new puzzle tactic patterns if
they are not labelled appropriately.

\subsection{Game Review}\label{evalS32}

An interesting unintended feature of this approach is that it can, of course,
predict puzzle themes for any chess position. Since inference time is so fast,
it is possible to iterate through any chess game's moves and analyse those
positions. For this example, a game played by the authors was analysed, and the
interesting positions (\Cref{alapin1,alapin2,alapin3,alapin4}), along with
their labels are discussed. Interestingly, the starting position
(\Cref{alapin1}) has the labels \texttt{crushing}, \texttt{hangingPiece}, and
\texttt{oneMove}. The model does correctly predict \texttt{opening}. 

After playing a strange sequence of pawn moves (\Cref{alapin2}), Black gets a
losing position, where White has a classic \emph{attack on f7} tactic. Indeed,
White's best move in this position is a \emph{bishop sacrifice},
\texttt{9.Bxf7+ Kxf7}, which walks into a knight check: \texttt{10.Ng5+ Ke8
11.Qxg4}. Right before the \emph{bishop sacrifice} was played, the model
correctly predicts \texttt{attackingF2F7}, \texttt{attraction} (the black king
is attracted onto a square where it can be checked by the knight) and
\texttt{sacrifice}. These themes appear only at this position, and nowhere else
in the game. As a side note, the position immediately prior to \texttt{10.Ng5+}
is labelled \texttt{discoveredAttack}. It is also at this point that the model
stops predicting \texttt{opening} and begins predicting \texttt{middlegame}.

This attack leads to what seems to be a mating position (\Cref{alapin3}).
Despite this, White does not have a checkmate. The player with the white pieces
testifies that he thought the position must have a forced mate, and the model
agrees, predicting \texttt{mate} and \texttt{mateIn1}. Both are wrong.

After many moves, White fails to convert his advantage (\Cref{alapin4}), and
ends up in an equal position. After the rook trade, \texttt{30...Rxe1+ 31.Rxe1
Bxd4}, the model switches its predictions from \texttt{middlegame} to
\texttt{endgame}. No more sharp positions appear for the rest of the game and
the model's predictions are equally uninteresting.

\begin{figure}[H]
  \begin{minipage}[t]{0.475\textwidth}
    \centering
    \chessboard[setfen=rnbqkbnr/pppppppp/8/8/8/8/PPPPPPPP/RNBQKBNR w KQkq - 0 1]
    \caption{The starting position.}
    \label{alapin1}
  \end{minipage}
  \hspace{0.05\textwidth}
  \begin{minipage}[t]{0.475\textwidth}
    \centering
    \chessboard[setfen=2rqkbnr/3npppp/p2p4/1ppB4/3PP1b1/2P2N2/PP3PPP/RNBQ1RK1 w k - 0 9]
    \caption{White has a winning advantage.}
    \label{alapin2}
  \end{minipage}
\end{figure}

\begin{figure}[H]
  \begin{minipage}[t]{0.475\textwidth}
    \centering
    \chessboard[setfen=2rqkb1r/4p1pp/pn1pQn2/1pp3N1/3PP3/2P5/PP3PPP/RNB2RK1 w - - 3 13]
    \caption{Despite looking lethal for Black, White does not have a forced mate.}
    \label{alapin3}
  \end{minipage}
  \hspace{0.05\textwidth}
  \begin{minipage}[t]{0.475\textwidth}
    \centering
    \chessboard[setfen=1r6/3kr1b1/3p3p/3n2p1/1p1P4/6N1/P2B1PPP/1R2R1K1 b - - 4 30]
    \caption{After the rook trade, White loses the \texttt{d4} pawn and loses the game soon after.}
    \label{alapin4}
  \end{minipage}
\end{figure}


\section{Tree-Based Approach}\label{evalS4}

\subsection{Strengths and Weaknesses}\label{evalS41}

The biggest weakness of the tree-based approach -- using $k$-NN to predict
labels and difficulty -- is the inference time. This has had large consequences
previously in the project, and this is an unavoidable result of using a
relatively expensive and non-vectorisable distance function. This downside
meant that the method could not be evaluated fully, as it was time prohibitive
to analyse the entire test set while using the entire training set.

Another downside is the sensitivity to parameters (\Cref{treeS13}). Different
parameters in the distance function define what it means for a puzzle to be
similar, and this, of course, can also vary drastically between chess players.
The other set of parameters that greatly influences the performance is the tree
generation parameters (\Cref{treeS12}). These are again non-trivial to
optimise, as it takes a long time to regenerate the meaningful search trees.
These parameters also change from person to person, especially the arbitrary
move consideration boundary, which was kept at 100 centipawns in this project.
It is very likely that a higher skilled player would not even consider
low-performing moves when comparing if two puzzles are similar, whereas a lower
rated player would calculate more variations deeper as he is unable to
understand the strength of a position with intuition alone.

However, within this weakness also lies the main strength of this method. This
tunability means chess players of all skills can benefit from the distance
function, as they can tweak it to their performance.

Of course, this also means this method of puzzle comparison is much more
explainable than the black-box deep learning approach. It is completely
possible to view a puzzle's meaningful move tree, and to reason about why a
certain puzzle is said to be close to another, or said to have a certain label
or difficulty.

\subsection{Puzzle Similarity}\label{evalS42}

Grouping puzzles (\Cref{treeS2}) and ranking puzzles by similarity
(\Cref{treeS21}) is unique to this method. Qualitative analysis of clustering
performance has been shown to have intriguing results (\Cref{treeS23}), as this
method was able to group together puzzles in ways that have not been
implemented before. The distance-based method of ranking puzzles by similarity
is also promising, and can likely be used to datamine game databases to find
similar positions. 


\chapter{Conclusion}\label{concChapter}

\section{Related Work}\label{concWork}

\subsection{A Language For Encoding Piece Relationships}

\citet{chessLanguage} describe a language to search across chess positions. The
main features of this language are descriptions for a chess piece
attacking/defending another, attacking/defending a square, being located at a
square, a square not being available for the enemy king, and the structure of
white/black pawns on the board.

With their novel language, they are able to search a chess database for a
pre-determined pattern, such as the \emph{Greek gift sacrifice}, defined in the
language as ``\texttt{kg8, pf7, pg7, B(ph7), Nf3, Qd1, Pe5}''. With inspiration
from the FEN notation, this string corresponds to a black king on g8
(\texttt{kg8}), black pawns on f7, g7; a white bishop attacking a black pawn on
h7 (\texttt{B(ph7)}), a white knight on f3 (ready to deliver a check with
\texttt{Ng5+}, a common motif in \emph{Greek gift sacrifices}), a white queen
on d1 (\texttt{Qd1}), and finally, a white pawn on e5 (\texttt{Pe5}),
dislodging the usual black knight on f6.

This returns positions such as the one in Figure \ref{chess3}. After
\texttt{16...Be7 17.O-O}, Black blundered with \texttt{17...Bxa3??}, after
which, a \emph{Greek gift sacrifice} (\texttt{18.Bxh7!}, shown in Figure
\ref{chess4}) was made, eventually leading to a win for White.

\begin{figure}[H]
    \begin{minipage}{0.475\textwidth}
        \centering
        \chessboard[setfen=r1b2rk1/qp3ppp/p1n1pb2/4P3/3P4/P1BB1N2/5PPP/1R1QK2R
        b K - 0 16]
        \caption{\textbf{Pirc, V -- Porreca, G}, YUG-ITA m 1953, move 16.}
        \label{chess3}
    \end{minipage}
    \hspace{0.05\textwidth}
    \begin{minipage}{0.475\textwidth}
        \centering
        \chessboard[setfen=r1b2rk1/qp3ppB/p1n1p3/4P3/3P4/b1B2N2/5PPP/1R1Q1RK1 b
        - - 0 18]
        \caption{\textbf{Pirc, V -- Porreca, G}, YUG-ITA m 1953, move 18. Black
        resigned after 6 moves.}
        \label{chess4}
    \end{minipage}
\end{figure}

Their language is also able to deal with some light variations, as it is able
to identify the games shown in Figures \ref{chess5}, \ref{chess6}. In both of
these positions, White has the amazing move \texttt{1.Qh6+!}, following with
\texttt{2.Rh8\#} if \texttt{1...Kxh6}, and either \texttt{2.Rf7\#} or
\texttt{2.Rb7+} (leading to a quick mate) if \texttt{1...gxh6}. 

This pattern, whilst very rare, is undeniably identical between the 2 games.
The unavailability of the \texttt{g6} square to the enemy king, combined with
the harmony of White's pieces leads to the same tactic in both games.

\begin{figure}[H]
    \begin{minipage}{0.475\textwidth}
        \centering
        \chessboard[setfen=2R5/4bppk/1p1p4/5R1P/4PQ2/5P2/r4q1P/7K w - - 5 50]
        \caption{\textbf{Carlsen, M -- Karjakin S}, World Chess Championship
        2016, move 50.}
        \label{chess5}
    \end{minipage}
    \hspace{0.05\textwidth}
    \begin{minipage}{0.475\textwidth}
        \centering
        \chessboard[setfen=5R2/bp4pk/2n3p1/P7/P1q3bP/6P1/3Q3K/1R6 w - - 1 32]
        \caption{\textbf{Popov, N -- Novopashin, A}, URS-ch otbor 1979, move
        32.}
        \label{chess6}
    \end{minipage}
\end{figure}

The work of \citet{chessLanguage} is a promising proof of concept that shows
the power of a language that allows to specify piece relationships on a more
abstract level than previously possible. The biggest drawback of their
solution, as mentioned by the authors, is the fact that this language still
requires an expert with pre-existing extensive knowledge to encode the tactics
into their language.

This work is quite similar to this project's tree-based puzzle analysis
(\Cref{treeChapter}). Tree-based puzzle analysis, with a distance function, is
very computationally expensive due to the need to analyse game positions with a
chess engine. The language developed by \citet{chessLanguage} is far more
performant, which makes it much more accessible. The downside, of course, is
the need for domain knowledge. 


\subsection{CQL: Chess Query Language}

The Chess Query Language (CQL), invented by \citet{cql}, is another
implementation of an advanced way to find chess positions in a given database.
Since its inception in 2004, it has grown and is able to support very powerful,
sometimes esoteric, queries to find predefined patterns.

An example of such a query is provided on the CQL website \citep{cqlSmothered},
and is shown in Figure \ref{cql} for reference. In this query, \texttt{btm}
means `black-to-move` and \texttt{mate} means checkmate is played. This
language is incredibly powerful and terse, as it allows specifying complicated
piece relationships and supports quality-of-life features such as matching
mirror positions or reversed-colour positions.

\begin{figure}[H]
    \centering
    \includegraphics[width=0.45\linewidth]{background/img/cql.png}
    \caption{A CQL query to find positions where smothered mate occured.}
    \label{cql}
\end{figure}

In addition to Costeff's CQL, there exists a from-scratch clone of CQL6
\citep{cqli} which includes extra features and supports other chess variants.

This is a very powerful tool, but suffers from the same drawback as the work of
\citet{chessLanguage}: it requires extensive knowledge to use effectively. It
has been shown by \citet{cql} that CQL can support incredibly niche and complex
tactical patterns. It is possible that an expert, armed with the unsupervised
clustering (\Cref{treeS2}) and this language, can create efficient queries to
find critical chess positions.

\subsection{The Chess Transformer}\label{chessTransformerSection}

\citet{chessTransformer} demonstrate the ability of transformers to learn the
rules of chess and complex gameplay by analysing PGN games with a fine-tuned
GPT-2 transformer. By treating PGN games as a sequence of natural language
words, the authors show successful results and are able to generate new games
without specifying chess rules.

A key downside of their work is that their model generates illegal moves, which
have to be filtered manually using a chess library \citep{chessTransformer}.
This `hallucination' effect is a downside of using transformers, and other
generative techniques, in chess. 

This work is very similar to this project's work with transformers
(\Cref{mlChapter}). The key difference is that the work by
\citet{chessTransformer} operates on sequences of chess moves, while the work
in this project uses a transformer encoder to learn relationships between the
chess squares in a static position. It is unlikely that adding the extra
complexity of learning PGN outweighs the benefits, especially given the task of
this project. However, along with this project's work on transformers, the work
of \citet{chessTransformer} shows that there is potential in applying the
transformer architecture to the logical structure of chess.

\section{Closing Remarks}

At the be

\section{Future Work}

%\input{appendix/appendix.tex}

\bibliographystyle{unsrtnat}
\bibliography{bibs/bibliography}

\end{document}
